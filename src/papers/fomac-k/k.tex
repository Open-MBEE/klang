\section{The K Language}
\label{sec:k}

\Klang{} is a textual language with various constructs for
modeling and programming. The \uml{} and \sysml{} communities use the term {\em modeling}, whereas the formal methods community normally uses the term {\em specification}. We consider these terms for equivalent in this context, and shall use them interchangeably. A model (specification) is an abstract representation
of a system (be it physical, conceptual, software, etc.), which has a concrete implementation, which in the software context is the program. It can be a physical
object, however. The primary intended use for \Klang{} is to easily create models and in the software context: implementations, and then be able to 
perform analysis on them. We primarily see \Klang{} being used by 
system modelers who are used to
expressing their models in SysML/UML. In this section, we briefly
provide an overview of the \Klang{} language. The presentation is centered
around the example \Klang{} model of geometric shapes, shown in Figure 
\ref{fig:shapes}.

\begin{figure}
\centering
\begin{tabular}{c}
\hline \\
\lstinputlisting{examples/shapes.k} \\ \\
\hline
\end{tabular}
\caption{A simple model of geometrical shapes expressed using \Klang{}}
\label{fig:shapes}
\end{figure}


\begin{description}

\item [Classes]: Similar to classes in other languages, the classes in
  \Klang{} provide the construct for performing abstraction. It is \Klang{}'s
   module concept. Classes can be arranged in packages, as in \java.
   A class can contain properties (corresponding to fields in \java), functions, 
   and constraints, as discussed below. For example, class \name{Triangle} 
   contains three properties, which are of type \name{TAngle}.

\item [Inheritance]: \Klang{} provides the \code{extends} keyword for
  specifying an inheritance relation. In Figure~\ref{fig:shapes},
  \name{TAngle} class extends the \name{Angle} class. As a result,
  \name{TAngle}, not only inherits the properties and functions of
  \name{Angle}, but also the constraints. \Klang{} also allows for
  multiple inheritance. Property and function names must be uniquely
  specified.

\item [Primitive Types]: \Klang{} provides the following primitive
  types: \code{Int}, \code{Real}, \code{Bool}, \code{String},
  \code{Char}, \code{Unit}.

\item [Collections]: \Klang{} provides \code{Set}, \code{Seq}, and
  \code{Bag} as the three basic collections. \Klang{} also provides
  support for \name{Tuple} to create Cartesian products. The shapes model
  does not contain collections.

\item [Properties]: In \Klang{}, properties can be present within
  classes or at the outermost level. Each properties must have a name
  and a type. Our use of the term {\em property} is due to the use of this term
  in the model-based engineering (\uml/\sysml) community for name-type pairs,
  which in programming language and formal methods terminology normally are called
  constant/variable/field declarations.
  In the model shown in Figure~\ref{fig:shapes}, class
  \name{Shape} contains a single property named \name{sides} of type
  \code{Int}.

\item [Modifiers]: Each property can also have one or more
  \emph{modifiers} specified for it, for example
  {\bf val}/{\bf var} to make the property read only or writable (the default being
    read only). The shapes model does not contain modifiers.
    % KH begin:
    % The following is not quite true. ordered is associated with a simple
    % property of the form x : A, and not with x : Bag[A]. It is too complicated
    % to explain here, perhaps, which is why I have commented it out.
    % KH end
    %\begin{description}
    %  \item [val/var] to make the property read only or writable (the default being
    % read only)
    %\item [ordered/unique] apply to collections to make them ordered
    %  and unique as needed. For example, an \code{ordered Bag} is the
    %  same as a \code{Seq} and a \code{unique Bag} is the same as a
    %  \code{Set}.
  %\end{description}
  
\item [Constraints]: \Klang{} provides syntax for specifying
  constraints in a class. This is done using the \code{req} keyword
  (we use the term {\em requirements} for constraints)
  followed by a name (optional), and an expression that specifies the
  constraint on the class. 
  For example, in class \name{Angle}, the
  \name{value} of any angle should always be between 0 and 360
  degrees. Multiple constraints can also be specified. Their effect is
  the same as if all expressions were conjoined into a single
  constraint. For example, each instance of the \name{Triangle} class
  should have exactly three sides and the sum of the angles should be
  exactly 180 degrees.
  
\item [Functions]: A function provides the ability to perform
  computation. In \Klang{}, functions can take arguments and return
  the result of the computation of the function. The \name{eq}
  function in class \name{Angle} compares the given angle's value to
  its own and returns a \code{Bool}.

\item [Function Specifications]: Each function in \Klang{} can also
  have a \emph{specification} associated with it. The specification
  can be a \code{pre}-condition and/or a
  \code{post}-condition. The shapes examples does not contain function
  specifications. They are discussed in more detail along with
  an example in Section~\ref{sec:change}.
  
\item [Expression Language]: Similar to other high level programming
  languages such as Java and Scala, \Klang{} provides a rich
  expression language for specifying behaviors, functions, and
  constraints. \Klang{} provides multiple operators such as
  implication, conjunction, disjunction, arithmetic operators,
  if-then-else statements, for/while loops, blocks, predicate logic
  with quantifiers etc. 

\item [Annotations] \Klang{} provides the ability to create new
  annotations by specifying a name and a type for the annotation. The
  annotations can then be applied by writing an $@$ sign followed by
  the annotation name and possible parameters, immediately before the 
  element that is desired to be annotated. There is no limit on how many 
  annotations can be applied to any entity. 

\item [Comments]: Single line comments can be specified with the
  prefix $--$ and multi-line comments are specified with $===$ as both
  the start and end of the multi-line comment.

\end{description}

In addition to the language constructs described here, \Klang{} also
has support for programming with \name{side effects} 
(assignment, sequential composition, and looping constructs), 
\name{type abbreviations}, as well as \sysml{}/\uml{} specific concepts such as \name{associations} and \name{multiplicities} (which can be used for specifying the allowed size of a property). 
To conserve space, these constructs are not discussed in further detail. 
Appendix~\ref{app:grammar} shows
part of the \Klang{} ANTLR grammar, omitting 
grammar rules dealing with parsing of values of primitive types, such as digits, strings, etc.

Currently, the \Klang{} infrastructure comes with a parser that has
been generated using ANTLR version 4. Using the parser, an AST is
created that is used for performing type checking. In addition to
performing type checking, we have also created a translation of
\Klang{} to SMT2, which currently is processed by the \zthree{} theorem prover.
This is used as a means to perform various checks
such as function specification satisfiability, class consistency
checking, and model generation. The entire \Klang{} infrastructure is
implemented using \scala.

 
