\section{The K Language}

\Klang{} is a textual language with various constructs for
programming, modeling, and creating specifications. The primary
intended use for \Klang{} is to easily create models and
specifications, and then be able to perform analysis on them. We
primarily see \Klang{} being used by system modelers who are used to
expressing their models in SysML/UML. In this section, we briefly
provide an overview of the \Klang{} language.

\begin{figure}
\centering
\begin{tabular}{c}
\hline \\
\lstinputlisting{examples/shapes.k} \\ \\
\hline
\end{tabular}
\caption{A simple model of geometrical shapes expressed using \Klang{}}
\label{fig:shapes}
\end{figure}


\begin{description}

\item [Comments]: Single line comments can be specified with the
  prefix $--$ and multi-line comments are specified with $===$ as both
  the start and end of the multi-line comment.

\item [Primitive Types]: \Klang{} provides the following primitive
  types: \code{Int}, \code{Real}, \code{Bool}, \code{String},
  \code{Char}, \code{Unit}

\item [Collections]: \Klang{} provides \code{Set}, \code{Seq}, and
  \code{Bag} as the three basic collections. \Klang{} also provides
  support for \name{Tuple} to create Cartesian products. 

\item [Properties]: In \Klang{}, properties can be present within
  classes or at the outermost level. Each properties must have a name
  and a type. In the model shown in Figure~\ref{fig:shapes}, class
  \name{Shape} contains a single property named \name{sides} of type
  \code{Int}.

\item [Modifiers]: Each property can also have one or more
  \emph{modifiers} specified for it:
  \begin{description}
    \item [val/var] to make the property read only or writable
    \item [ordered/unique] apply to collections to make them ordered
      and unique as needed. For example, an \code{ordered Bag} is the
      same as a \code{Seq} and a \code{unique Bag} is the same as a
      \code{Set}.
  \end{description}

\item [Classes]: similar to classes in other languages, the classes in
  \Klang{} provide the simplest way of performing abstraction. A class
  can contain properties and functions. For example, class
  \name{Triangle} contains three properties, which are of type
  \name{TAngle}.
  
\item [Class Constraints]: \Klang{} provides syntax for specifying
  constraints in a class. This is done using the \code{req} keyword
  followed by a name (optional), and an expression that specifies the
  constraint on the class. For example, in class \name{Angle}, the
  \name{value} of any angle should always be between 0 and 360
  degrees. Multiple constraints can also be specified. Their effect is
  the same as if all expressions were conjoined into a single
  constraint. For example, each instance of the \name{Triangle} class
  should have exactly three sides and the sum of the angles should be
  exactly 180 degrees.
  
\item [Functions]: A function provides the ability to perform
  computation. In \Klang{}, functions can take arguments and return
  the result of the computation of the function. The \name{eq}
  function in class \name{Angle} compares the given angle's value to
  its own and returns a \code{Bool}.

\item [Function Specifications]: Each function in \Klang{} can also
  have a \emph{specification} associated with it. The specification
  can be either a \code{pre}-condition or a
  \code{post}-condition. These are discussed in more detail along with
  an example in Section~\ref{sec:change}.
  
\item [Inheritance]: \Klang{} provides the \code{extends} keyword for
  specifying an inheritance relation. In Figure~\ref{fig:shapes},
  \name{TAngle} class extends the \name{Angle} class. As a result,
  \name{TAngle}, not only inherits the properties and functions of
  \name{Angle}, but also the constraints. \Klang{} also allows for
  multiple inheritance. Property and function names must be uniquely
  specified.

\item [Expression Language]: Similar to other high level programming
  languages such as Java and Scala, \Klang{} provides a rich
  expression language for specifying behaviors, functions, and
  constraints. \Klang{} provides multiple operators such as
  implication, conjunction, disjunction, arithmetic operators,
  if-then-else statements, for/while loops, blocks, predicate logic
  with quantifiers etc. 

\item [Annotations] \Klang{} provides the ability to create new
  annotations by specifying a name and a type for the annotation. The
  annotations can then be applied by writing an $@$ sign followed by
  the annotation name immediately before the element that is desired
  to be annotated. There is no limit on how many annotations can be
  applied to any entity. 

\end{description}

In addition to the language constructs described here, \Klang{} also
has support for \name{packages}, \name{side effects}, \name{types},
\name{associations} -- SysML/UML specific concept, and
\name{multiplicities}, which can be used for specifying the allowed
size of a property. To conserve space, these constructs are not
discussed in further detail.

Currently, the \Klang{} infrastructure comes with a parser that has
been generated using ANTLR version 4. Using the parser, an AST is
created that is used for performing type checking. In addition to
performing type checking, we have also created a translation of
\Klang{} to SMT2. This is used as a means to perform various checks
such as function specification satisfiability, class consistency
checking, and model generation. The entire \Klang{} infrastructure is
implemented using Scala.

 
