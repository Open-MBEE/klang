
\section{Introduction}

The core topic of this paper is the concept of {\em change}, and how
it relates to the way we {\em model} as well as {\em program} our
systems, and how we can ensure correctness of change using modern
verification technology.  We shall specifically discuss this topic by
introducing the wide-spectrum modeling and programming language
\Klang, under development at NASA's Jet Propulsion Laboratory (JPL),
and demonstrate change scenarios and their verification in \Klang.

The first call for opinion statements on the topic of change, with
this journal in mind, was published in \cite{steffen-isola-2014}, in
which it is characterized as ``{\em a discipline for rigorously
  dealing with the nature of today's agile system development, which
  is characterized by unclear premises, unforeseen change, and the
  need for fast reaction, in a context of hard to control frame
  conditions, like third party components, network-problem, and
  attacks}''.
%
Our view is that change fundamentally can be considered as a software
verification problem, where the question is the following: given a
program $P_1$, and a new program $P_2$, does $P_2$ implement/refine
$P_1$?  As such, in the extreme, the topic of correctness under change
can potentially be considered as the well known topic of correctness.

As is well known, analysis of correctness can be performed dynamically
by analyzing execution traces. Dynamic methods include such topics as
testing, runtime verification, and specification mining, where learned
models of previous behavior is compared to models of current behavior
(after change). Dynamic methods are known to scale well, which is
their attraction, but are also known to yield false negatives: not to
report errors that exist, and in some cases (i.e. runtime analysis of
data races and deadlocks) to yield false positives (to report errors
that do not exist). Our focus in this paper is on static analysis of
code: where proofs are carried out on the basis of the structure of
the code.

The \Klang{} language is being developed at NASA's Jet Propulsion
Laboratory (JPL), as part of a larger effort at JPL to develop an
in-house systems modeling language and associated tool set, referred
to as \ems{} (Engineering Modeling System). \ems{} is based on the
graphical \sysml{} formalism \cite{sysml}, which is a variant of
\uml{} \cite{uml}, both designed by the OMG (Object Management Group)
technology standards consortium. \sysml{} is meant for systems
development more broadly considered, including physical systems as
well as software systems, in contrast to \uml{}, which is mainly meant
for software development.  \ems{} is being developed (and already in
use) to support the design of NASA's proposed mission to Jupiter's
Moon Europa, referred to as the {\em Europa Clipper mission}
\cite{europa-clipper}, planned to launch around 2022 at the moment of
writing.

The \Klang{} language design was initially triggered by a desire from
within the \ems{} project to create a textual version of
\sysml. Although the graphical nature of \sysml{} can be attractive
from a readability point of view, it suffers from a number of issues,
including lack of clear semantics, lack of analysis capabilities,
requiring heavy mouse-movement and clicking, and generally demanding
much more visual space than a textual formalism. The \Klang{} language
is in addition inspired by the idea of combining modeling and
programming into one language. This idea has been, and is being,
pursued by others in various forms, including past works on
wide-spectrum specification languages. See the related work section
(Section \ref{sec:related-work}) for a more detailed discussion. It is
interesting to observe, that modern programming languages to an
increasing degree look and feel like specification languages from the
past.

A constantly returning discussion point in the design of \Klang{} has
been whether to actually design a new language or adopt an existing
specification or programming language. The decision to design a new
language was in part influenced by the felt need (perhaps unwarranted)
to control the syntax and the tool development stack, including
parsers, compilers, etc. Whether this decision was/is correct, an
important factor that makes \Klang{} interesting is that it is being
designed to meet the needs of a real space mission. As such it can be
considered as a confirmation of already existing work in the area of
specification as well as programming languages.

The contents of the paper is as follows. 
Section \ref{sec:k} introduces the \Klang{} language, focusing on a small
analyzable subset, which is currently being tested during the initial 
design phase of the Europa
Clipper mission.
Section \ref{sec:change} discusses various aspects of change by formalizing
them in \Klang.
Section \ref{sec:related-work} outlines related work, and 
Section \ref{sec:conclusion} concludes the paper.
Appendix \ref{app:grammar} contains the grammar for \Klang{} in \antlr{} 
\cite{antlr} format.
