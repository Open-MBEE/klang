
\section{Related Work}
\label{sec:related-work}

\Klang{} is intended to represent a textual modeling language capable
of representing \sysml{} concepts, specifically class diagrams with
constraints.  However, as mentioned in the introduction, it also
contains programming constructs.  As such it can be perceived as a
wide-spectrum modeling/programming language.

Wide spectrum specification languages have been investigated to length
in the formal methods community. One of the well-known examples
is \vdm{} \cite{vdm78,bjoerner-jones-82,jones90,jones-shaw-90}. \vdm{}
in its original form \cite{vdm78} provided a combination of procedural
programming and functional programming and specification using sets,
lists and maps (with proper mathematical notation), and higher-order
predicate logic. \vdmpp{} \cite{vdmplusplus05} added
object-orientation to \vdm{}, which is now part of the \vdm{}
standard. The \raiselang{} specification language (\rsl{})
\cite{raise92} is a wide-spectrum language taking inspiration from
\vdm{} as well as from other modeling languages such as \zlang{}
\cite{spivey-Z-1988}, and from algebraic equational specification
languages. Here refinement is the simpler theory implication: the implementation
shall imply the specification in a logic sense. \asml{} \cite{asml05} is a 
more recent wide-spectrum specification language, in many ways similar to \vdm{}, 
but based on the fundamental concept that operations operate on algebras. 
Other fundamental works on refinement include (not a comprehensive list): \cite{wirth-refinement-71,hoare-sanders-refinement-86,morgan-refinement-94,woodcock-sanders-z-96,back-wright-refinement-98,abrial-eventb-10}.

\alloy{} \cite{jackson-alloy-12} added new life to this community by being 
supported by a automated SAT solver. In many respects, \Klang{} is close in spirit 
to \alloy{}, but differs by being supported by an automated SMT solver 
(in contrast to a SAT solver), resulting in a richer set of constructs, including arithmetic, being exposed to analysis. \Klang{} also combines a type view as found in traditional specification and programming languages, as well as a relational 
view, whereas \alloy{} is purely relational. We are of the belief that the notion
of a type is fundamental to programming as well as to modeling. In contrast
to automated provers, interactive theorem provers such as
\pvs{} \cite{cade92-pvs,pvs-website}, \coq \cite{coq-website}, and 
\isabelle \cite{isabelle-website}, allow the user to steer the proofs. 
Although this allows to perform more complex proofs, it is also requires more
skills of the user, and time, which is often a limited resource in software development projects.

Several high-level programming languages have been developed in recent
years, including the early \sml{} (Standard ML) \cite{standard-ml-97},
its derivative \ocaml{} \cite{ocaml}, and \haskell{}
\cite{haskell}. However, also \java{} can be considered high-level due
to its libraries of collections (sets, lists, and maps), as well as
the iterator concept. \python{} \cite{python} is close to combining
object-oriented and functional programming. The \scala{} \cite{scala}
language does this to the full extent, as does the \fortress{}
\cite{fortress}. The close relationship between \scala{} and \vdm{} is discussed in
\cite{havelund-scala-vdm-12}.

Specification constructs have been introduced in programming
languages, in the form of design-by-contract (pre/post conditions +
class invariants). Examples are \eiffel{} \cite{eiffel} and
\specsharp{} \cite{specsharp}, where contracts are part of the
language. \scala{} has library functions for writing pre/post
conditions on functional programs \cite{odersky-rv10}. Finally, The
\jml{} language \cite{jml} allows to write design-by-contract
specifications for \java{} as comments. These are ignored by the
standard \java{} compiler, and therefore must be processed with
special tools. \eml{} (Extended ML) \cite{sannella-eml-97} takes a
slightly different approach to specification and formal development of
\sml{} programs.  \eml{} specifications look just like \sml{} programs
except that axioms are allowed in signatures and in place of code in
structures and functors. Some \eml{} specifications are executable,
since \sml{} function definitions are just axioms of a certain special
form. This makes \eml{} a wide-spectrum language which can be used to
express every stage in the development of a \sml{} program from the
initial high-level specification to the final program itself and
including intermediate stages in which specification and program are
intermingled.

Programming languages are now also being built with verification in mind.
\dafny{} \cite{leino-lpar-2010} supports specifications that can be used to write functional-correctness conditions for programs.  It is supported by verifier, which is implemented on top of the \boogie{} verification engine, which itself is built on top of \zthree. 
\whythree{} \cite{filliatre-why3-2011} provides a rich language for specification and programming, called \whyml{}, and relies on external theorem provers, both automated and interactive, to discharge verification conditions. A user can write \whyml{} programs directly and get correct-by-construction \ocaml{} programs through an automated extraction mechanism. 

Apart from all the work in the language domain, performing analysis
and verification of languages has also seen great
strides.~\cite{holzmann-spin-2004} performs model checking of models
expressed in Promela and Java
PathFinder~\cite{havelund-jpf-00,havelund-visser02} performs model
checking of Java programs. The latter can be used not only for
expressing models in Java and verifying them, but also for proving
program properties.~\cite{ball2010slam2} presents results of applying
static analysis and counter-example guided abstraction refinement to
device drivers in a large scale industry setting. 

The great improvements in model checking, static analysis, theorem
proving, and SMT solvers such as Z3~\cite{de2008z3} have all
contributed to investigating and dealing with software change. To this
effect, differential symbolic execution~\cite{person2008differential}
has been investigated for establishing equivalence between two
versions of a program.~\cite{lahiri2012symdiff} uses verification
conditions and SMT solvers for detecting semantic change between two
closely related versions of a function (program) by discovering inputs
to a function that cause the outputs to
change.~\cite{godlin2009regression} deals with \emph{regression
  verification} and present a technique for doing equivalence checking
of C programs, by using the older version of the program almost as a
\emph{specification} for the new version of the program. A large part
of the inspiration for such work comes from the theorem proving
community.



