
\section{Related Work}
\label{sec:related-work}

\Klang{} is intended to represent a textual modeling language capable of representing \sysml{} concepts, specifically class diagrams with constraints.
However, as mentioned in the introduction, it also contains programming constructs.
As such it can be perceived as a wide-spectrum modeling/programming language.

Wide spectrum specification languages have been investigated to length in the formal 
methods community. One of the most well-known examples is \vdm{} 
\cite{vdm78,bjoerner-jones-82,jones90,jones-shaw-90}. \vdm{} in its
original form \cite{vdm78} provided a combination of procedural programming and
functional programming and specification using sets, lists and maps (with proper 
mathematical notation), and higher-order predicate logic. \vdmpp{} 
\cite{vdmplusplus05} added object-orientation to \vdm{}, which is now part of
the \vdm{} standard \cite{vdmsl}. The \raiselang{} specification language (\rsl{})
\cite{raise92} is a wide-spectrum language taking inspiration from \vdm{} as well as 
from other modeling languages such as \zlang{} and from algebraic specification 
languages, such as \clear{}. \asml{} \cite{asml05} is a more recent wide-spectrum
specification language, in many ways similar to \vdm{}, but based on the fundamental concept that operations operate on algebras.

Several high-level programming languages have been developed in recent years,
including  the early \sml{} (Standard ML) \cite{standard-ml-97}, its derivative
\ocaml{} \cite{ocaml}, and \haskell{} \cite{haskell}. However, also \java{}
can be considered high-level due to its libraries of collections (sets, lists, and 
maps), as well as the iterator concept. \python{} \cite{python} is close to 
combining object-oriented and functional programming. The 
\scala{} \cite{scala} language does this to the full extent, as does the
\fortress{} \cite{fortress}.

Specification constructs have been introduced in programming languages, in the form
of design-by-contract (pre/post conditions + class invariants). Examples are
\eiffel{} \cite{eiffel} and \specsharp{} \cite{specsharp}, where contracts are 
part of the language. \scala{} has library functions for writing pre/post conditions 
on functional programs \cite{odersky-rv10}. Finally, The \jml{} language \cite{jml} 
allows to write design-by-contract specifications for \java{} as comments. These are 
ignored by the standard \java{} compiler, and therefore must be processed with 
special tools. \eml{} (Extended ML) \cite{sannella-eml-97} takes a slightly different approach to specification and formal development of \sml{} programs.
\eml{} specifications look just like \sml{} programs except that axioms are allowed in signatures and in place of code in structures and functors. Some \eml{} specifications are executable, since \sml{} function definitions are just axioms of a certain special form. This makes EML a wide-spectrum language which can be used to express every stage in the development of a \sml{} program from the initial high-level specification to the final program itself and including intermediate stages in which specification and program are intermingled.

\notethis{References for Z and Clear.}.

\notethis{Other relevant literature, like C verification tools, diff tools, ...}

\notethis{What we do different.}


  
  
Model checking software:
  Java PathFinder 
    \cite{havelund-jpf-00}
    \cite{havelund-visser02}  
  \cite{holzmann-spin-2004}
