\documentclass{llncs}

%\title{Combining Modeling and Programming\thanks{
\title{Verified Change\thanks{
  The work described in this publication was carried out at Jet Propulsion
  Laboratory, California Institute of Technology, under a contract with
  the National Aeronautics and Space Administration.}}
\author{Klaus Havelund \and Rahul Kumar}
\institute{
  Jet Propulsion Laboratory\\
  California Institute of Technology\\
  California, USA
}


\usepackage{multirow}
\usepackage{multicol}
%\usepackage[T1]{fontenc}
%\usepackage[latin9]{inputenc}
%\usepackage{babel}
%\usepackage{alltt}
%\usepackage{comment}
%\usepackage{multirow}
%\usepackage{graphicx}

% -----------------
% --- comments: ---
% -----------------

\newcommand{\notethis}[1]{
  \color{red}
  \vspace{0.3cm}
  \noindent\makebox[\linewidth]{\rule{\paperwidth}{0.4pt}}\\
  {\sc NOTE - } #1\\
  \noindent\makebox[\linewidth]{\rule{\paperwidth}{0.4pt}}\\
  \vspace{0.3cm}
  \color{black}
}

% ------------------
% --- listings: ---
% ------------------

\usepackage{url}
\usepackage{listings}
\usepackage{color}
\usepackage{amssymb}
\usepackage{framed}
\usepackage{graphicx} 
\usepackage{caption} 
\captionsetup{compatibility=false}
\usepackage{subcaption}
\usepackage{fancyvrb}
\usepackage{verbatimbox}
\usepackage{verbatim}
\usepackage{fancyvrb}
\usepackage{apalike}
\usepackage{SCITEPRESS}

\definecolor{light-gray}{gray}{0.95}

\lstdefinelanguage{K}{
  morekeywords={package,import,class,unique,extends,with,match,new,if,then,else,while,for,yield,val,var,this,case,type,implicit,private,protected,abstract,final, req, Int, String, Bool, fun, pre, post},
  otherkeywords={&,\{,\},_*},
  literate=
      {\#\#}{{$\oplus\;$}}2
      {|->}{{$\longmapsto\;$}}2
      {-->}{{$\Longrightarrow\;$}}2
      {->}{{$\rightarrow\;$}}2
      {~}{{$\sim\;$}}2
      {>>}{{$\gg$}}2
      {++}{{$++$}}2
      {|==}{{$\models$}}2
  ,
  sensitive=true, 
  morecomment=[l]{//}, 
  morecomment=[s]{/*}{*/},
  stringstyle=\ttfamily,
  numbers=left,   
  numberstyle=\footnotesize, 
}

\lstdefinelanguage{SMT}{
  morekeywords={define,declare,sort,const,fun,datatypes,assert,forall,exists,and,or,ite,select,Int,Real,Bool,Array},
  sensitive=true, 
  morecomment=[l]{;}, 
  stringstyle=\ttfamily,
%  numbers=none,   
%  numberstyle=\footnotesize, 
}

 
% --- Referring to code: ---

\newcommand{\name}[1]{{\em #1}}
%\newcommand{\code}[1]{\texttt{#1}}
\newcommand{\issue}[2]{{\bf ** #1: #2}}

\newcommand{\Klang}{{K}}
\newcommand{\tracecontract}{{TraceContract}}
\newcommand{\logfire}{{LogFire}}
\newcommand{\vdm}{VDM}
\newcommand{\vdmpp}{\vdm$^{++}$}
\newcommand{\asml}{Asml}
\newcommand{\raiselang}{Raise}
\newcommand{\rsl}{RSL}
\newcommand{\zlang}{Z}
\newcommand{\alloy}{Alloy}
\newcommand{\clear}{Clear}
\newcommand{\eiffel}{Eiffel}
\newcommand{\dafny}{Dafny}
\newcommand{\whythree}{Why3}
\newcommand{\whyml}{WhyML}
\newcommand{\eml}{EML}
\newcommand{\specsharp}{SpeC$^{\#}$}
\newcommand{\java}{Java}
\newcommand{\clang}{C}
\newcommand{\scala}{Scala}
\newcommand{\sml}{SML}
\newcommand{\ml}{ML}
\newcommand{\haskell}{Haskell}
\newcommand{\fortress}{Fortress}
\newcommand{\python}{Python}
\newcommand{\ocaml}{Ocaml}
\newcommand{\jml}{JML}
\newcommand{\uml}{UML}
\newcommand{\sysml}{SysML}
\newcommand{\ltl}{LTL}
\newcommand{\pvs}{PVS}
\newcommand{\isabelle}{Isabelle}
\newcommand{\coq}{Coq}
\newcommand{\boogie}{Boogie}
\newcommand{\zthree}{Z3}
\newcommand{\ems}{EMS}
\newcommand{\antlr}{ANTLR}
\newcommand{\slam}{SLAM}
\newcommand{\spin}{SPIN}
\newcommand{\promela}{Promela}{

\newcommand*{\scaleFactor}{0.75}

% \lstset{language=K,numbers=left,numberstyle=\scriptsize,numbersep=-21pt}
\lstset{language=K}


\begin{document}

\maketitle


\begin{abstract}

The formal methods community has over the years proposed various 
formally founded specification languages based on predicate logic 
and set theory. At the same time the model-based engineering community 
has proposed less formally founded graphical formalisms such as UML and 
SysML. We report on an effort to formally ground SysML in a textual 
formal language, named K, supporting classes, multiple inheritance, predicate 
logic and set theory. K contains programming constructs, and can thus be 
considered as a  wide-spectrum modeling and programming language. We further 
explain the translation of a subset of this textual language to the input 
language of the SMT-LIB standard, and the application of Z3 for analysis 
of the generated SMT-LIB formulas. The entire effort is part of a larger 
effort to develop a general purpose SysML development framework for designing 
systems, currently supporting the design of NASA's planned 2022 mission to 
Jupiter's Moon Europa. 

\end{abstract}


\keywords{Modeling, programming, constraints, refinement,
  verification, SMT, analysis.}


%%% TABLE OF CONTENTS - UNDER DEVELOPMENT: %%%

\vspace{1cm}
\color{red}
\noindent\makebox[\linewidth]{\rule{\paperwidth}{0.4pt}}\\

{\bf Table of Contents}

\begin{enumerate}
  \item Introduction: Discuss what 'change' means, and how it is the
    same idea as 'refinement'. Discuss 'modeling' and 'programming',
    and how they are becoming one. Our contributions. 
  \item The K Language: Present the K language. Discuss motivation,
    influences, syntax, ability and provide an example. Integration
    with Graphical Modeling (???). Verification with K.
  \item Change. What is it? How does one deal with it? What is our
    idea of dealing with change? Start presenting examples and
    discussing how they can be verified using K. For each example,
    show the \emph{old} and \emph{new} version of the program, discuss
    the differences, discuss what K does and what can be inferred from
    that.
  \item Related Work
  \item Conclusion
\end{enumerate}

\noindent\makebox[\linewidth]{\rule{\paperwidth}{0.4pt}}\\
\color{black}
\vspace{1cm}

%%%%%% 

\section{Introduction}
\label{sec:introduction}

Modeling is the activity of formulating an abstract description of a
system to be implemented (or possibly already implemented). Modeling
includes such activities as requirements capture in the initial phases
and design of higher-level architectural decisions in later
stages. Modeling has been studied by various communities, of which at
least two can be identified: the model-based engineering community and
the formal methods community. The {\em model-based engineering}
community has suggested graphical modeling languages such as UML
\cite{uml} and SysML \cite{sysml}, a variant of UML.  Both have been
designed by the OMG (Object Management Group) technology standards
consortium. SysML is meant for systems development more broadly
considered, including physical systems as well as software systems, in
contrast to UML, which is mainly meant for software development. These
graphical formalisms have received a high degree of popularity in
industry due to their two dimensional format, also sometimes referred
to popularly as {\em boxology}: boxes and arrows. However, drawbacks
of these formalisms include lack of precise semantics, lack of
analysis capabilities, tedious GUI operations, requiring lots
of visual real estate even for simple models, as well as large volumes
of technologies. Learning UML and SysML is not just learning very
large languages, it is also learning a large set of additional tools
needed to work with models. We formulate the hypothesis that some of these
drawbacks in part are due to the lack of a simple textual language, at
a size comparable to a programming language, underlying the
graphical notations.

From an even earlier point in time, since the 1960s, the {\em
  formal methods} community, part of the computer science community,
and closely connected to the programming language community, has
proposed numerous formally founded specification languages. Several of
these are based on predicate logic and set theory. These languages
are, compared to UML and SysML, concise, small, well defined in the
form of semantics, and in recent time well supported with analysis
capabilities. The obvious observation is that it might be fruitful to
study the interaction between the two classes of formalisms. Consider
furthermore that programming languages are gaining in abstraction,
such as for example combining object-oriented and functional
programming. An example is Scala, which has many commonalities with very early formal
specification languages, such as for example VDM \cite{vdm78}, and
specifically its object-oriented variant VDM$^{++}$
\cite{vdmplusplus05}.  The study should therefore include the
interaction between the graphical formalisms, the formal methods
modeling languages, and programming languages.

We report on an effort to formally ground SysML in a textual
formal specification language, named K (Kernel language), designed
specifically for this purpose.  
Our initial focus is on specifying and analyzing class diagrams
with constraints. K supports object-oriented concepts
such as classes, multiple inheritance, and object instances. The
contents of classes can be typed values, including functions, and
constraints over these expressed in higher-order predicate logic. 
K also contains programming constructs such as
variables, assignment statements, and looping constructs, and can as
such be seen as a wide-spectrum modeling and programming language.
The K language can be seen as a vehicle for giving semantics to SysML,
providing analysis capabilities, and even provide an alternative to
modeling with the mouse: writing textual models in K directly, just
like one normally writes programs.  We adhere to the school of thought
that modeling can be seen as programming in a language where some parts of the
model (program) at any point in time are executable, and some maybe
are not (yet).

The idea of merging modeling and programming in one language has been
suggested before, as will be discussed in the related work section.
Although K is not much different from previously suggested formalisms,  
our contribution is the creation of \Klang{} specifically in support of a
SysML model engineering tool set under development, to be used by
designers of the proposed 2022 mission to Jupiter's Moon Europa, also
referred to as the Europa Clipper mission concept
\cite{europa-clipper}.  The resulting tool set will support graphical
SysML modeling using MagicDraw \cite{magicdraw}, as well as
browser-based model viewing and editing, including of textual K
models. A first-order subset of K is furthermore translated to the input language of the
SMT-LIB standard, and currently processed with the Z3 SMT solver for
proving satisfiability of class definitions (are the constraints
consistent?), and model finding (find variable assignments satisfying
constraints, for example used in task scheduling).  The contribution here is
the handling of (multiple) class inheritance, which is typically not supported
by similar languages translated to SMT-LIB, as well as the allowance
of recursive class definitions. Multiple inheritance is a crucial part
of SysML, and therefore of K.

%The idea of merging modeling and programming in one language has been
%studied before, as will be discussed in the related work section. Our
%contribution is the development of \Klang{}, along with the technology
%to automatically translate it to SMT (for proving satisfiability) in
%the presence of multiple inheritance and constraints. This technology
%as part of a larger effort at NASA's Jet Propulsion Laboratory (JPL)
%to develop a SysML model engineering tool set to be used by designers
%of the proposed 2022 mission to Jupiter's Moon Europa, also referred to
%as the Europa Clipper mission concept \cite{europa-clipper}. This tool set,
%named EMS (Engineering Modeling System), supports graphical SysML
%modeling using MagicDraw \cite{magicdraw} as well as textual K
%modeling using any text editor of convenience. 

The paper is organized as follows. Section \ref{sec:k-syntax}
introduces a subset of the K language through an example, which is
similar to examples typically used to illustrate such formal
specification languages. Section \ref{sec:k2smt} outlines the
translation from K to the SMT-LIB input language for the purpose of
analysis of K models. This section is based on a different example
illustrating how K is actually currently used at JPL. Section
\ref{sec:usage} explains the integration of K within the SysML
development framework, as well as the usage of this. Section
\ref{sec:related-work} discusses related work, and finally Section
\ref{sec:conclusion} concludes the paper. Appendix \ref{app:grammar}
contains the grammar for K in ANTLR \cite{antlr} format.


\section{INTRODUCTION TO \Klang{}}
\label{sec:k-syntax}

In this section we introduce the \Klang{} language. We use the
\Klang{} model in Figure~\ref{fig:fs} as our running example for
discussing core concepts in \Klang{}. The example shows a model of a
file system modeled using \Klang{}. It is intended to be a basis for
discussing language features, and not a complete model of a file
system.

\begin{figure}
  \centering
  \begin{tabular}{c}
    %\hline \\
    \small
    \lstinputlisting{examples/fs.k}
    % \hline
    \end{tabular}
  \vspace{0.2cm}
  \caption{A simple model of a file system using \Klang{}.}
  \label{fig:fs}
\end{figure}
  
\Klang{} is a high level textual language which supports multiple
paradigms. It allows one to create \name{packages}, which are
collections of \name{classes}. Packages can be \name{imported} by
other \Klang{} files. Line 1 in Figure~\ref{fig:fs} shows an example
of a package declaration. Classes, as in other object-oriented
languages, provide a means for abstracting and grouping
properties (variables). In \Klang{}, classes may contain properties,
functions (there is no distinction between functions and methods), and
constraints (requirements). Scoping rules in \Klang{} are similar to
languages such as Java and C++. Lines 9 -- 12 in Figure~\ref{fig:fs}
declare an \code{Entry} class, which contains two members: property
\code{name} of type \code{String}, and function \code{size} that takes
no arguments and returns an \code{Int}. The function implementation is
not specified for function \code{size}. \code{String} is one of the
six primitive types provided by \Klang{}: \code{Int}, \code{Real},
\code{String}, \code{Char}, \code{Unit}, and \code{Boolean}. \Klang{}
also provides the following collections:

\begin{description}
\item [Bag:] collection of items not subject to any order
  or uniqueness constraints.
\item [Seq:] collection of items subject to an ordering, but
  no uniqueness constraints.
\item [Set:] collection of items subject to uniqueness, but no
  ordering constraints.
\item [OSet:] collection of items subject to uniqueness, as well as
  ordering constraints.
\end{description}

\noindent \Klang{} provides {\em predicate subtypes}. Line 5 specifies
a subtype named \code{Byte}, which is derived from the \code{Int} type
but restricted to values between 0 and 256. \Klang{} allows classes to
inherit from one or more classes. For example, class \code{Dir},
specified on lines 14 -- 20 extends the \code{Entry} class. As with
other languages, inheritance causes the child classes to inherit the
instance variables and functions of the parent classes, but in
addition, in \Klang{}, the child classes also inherit the constraints
from the parent classes. In the case of multiple inheritance, \Klang{}
requires that the property names be unique across all
classes. Functions on the other hand may be overloaded by changing the
function signature. Both class \code{File} and \code{Dir} inherit from
class \code{Entry}. 
Line 15 declares the variable \code{contents} using the keyword {\bf var},
indicating that this variable is mutable (can change value). Variables introduced
without this keyword (or with the keyword {\bf val}) are constants.
Lines 16 -- 19 in Figure~\ref{fig:fs} show the
implementation for the \code{size} function in the \code{Dir}
class. It makes use of the \code{sum} function, that is provided by
\Klang{} for all numerical collections. The \code{size} function is
the same as declared in class \code{Entry}. Currently, function bodies
cannot be declared more than once along an inheritance path. Functions
may take an arbitrary number of arguments and return a single
value. \Klang{} also provides tuples to group objects together. On
line 32, we see a constraint being specified for class \code{Block}
using the \code{req} (require) keyword. The constraint specifies that
the size function of \code{Block} should always return a value that is
less than or equal to the value specified in the global property
\code{SIZE\_OF\_BLOCK} (left unspecified). Any number of constraints
can be specified at the global scope or within classes. Constraints
are Boolean expressions, that restrict the values variables can
take. Constraints in a class can be considered class invariants.

Class \code{FS} (for \code{FileSystem}) contains two functions:
\code{mkDir} and \code{rmDir}. The \code{mkDir} function takes a
single argument (\code{n} of type \code{String}) and returns a
\code{FS} object which contains one additional directory entry that
has name \code{n}. The \code{rmDir} function has no body
specified. Both functions are defined along with a {\em function
  specification}. Function specifications are a list of {\em pre} and
{\em post} conditions that describe the precondition and postcondition
of the function. Any number of specifications may be provided. Line 39
specifies the precondition for function \code{mkDir} with the use of
an {\em existential} quantifier. It specifies that when creating a
directory in the file system, the given name \code{n} should not exist
in the current set of entries in the file system. \Klang{} provides
both {\em existential} and {\em universal} quantification in its
expression language. For the same function, line 42 specifies the
postcondition. \code{\$result} is a reserved word that refers to the
return value of the function. It can only be used when specifying
postconditions. The post condition for \code{mkDir} specifies that
function \code{mkDir} returns a \code{FS} object that has the same
size as the current \code{FS} object, which was used to create the new
directory. Lines 44 -- 51, the body of function \code{mkDir}, 
form a block consisiting of the declaration of two constants: \code{newDir} 
and \code{nc}, followed, in line 48, by the creation (and return) of
a new \code{FS} object by calling the constructor for class
\code{FS}. 
Note that the entries of a block are not separated by 
semicolon (`\code{;}'). In fact, K does not have a semicolon (nor newline) 
as statement separator, as for example seen in the programming language
Scala. The only argument provided to the \code{FS} constructor is a
\code{Dir} object which contains one additional \code{Dir} entry whose
name is \code{n}. \Klang{} provides constructors automatically for all
classes where the arguments are {\em named arguments}. Each named
argument is of the form `\code{member :: value}' where the `\code{::}'
notation is used as a form of assignment. Multiple named arguments can
be provided as a comma delimited list. It is not necessary to specify
a value for all members of a class. Any members that are specified in
a constructor call are assigned the specified value, and the rest are
left underspecified.  
%Line 48 invokes the \code{FS} constructor as
%well as the \code{Dir} constructor to create a new \code{FS} object
%that contains an additional directory. 
Function \code{rmDir} is
specified with no body, but only function specifications. The function
specifications require that function \code{rmDir} only execute if the
provided directory \code{n} exists in the current object's
contents. The postcondition specifies that the resulting \code{FS}
object should be either the same size or smaller relative to the
current object.

Expressions in \Klang{} are the core of the language. Expressions in
\Klang{} allow one to write assignment statements (side-effects),
binary expressions (such as and, or, implication, iff etc.), logical
negation, arithmetic negation, quantification, {\bf is} for checking
type, and {\bf as} for type casting. Any expression can make use of
other defined constructs such as variables, function application,
lambda functions, and dot expressions. \Klang{} also supports control
expressions such as \code{if-then-else}, \code{while}, \code{match},
\code{for}, \code{continue}, \code{break}, and \code{return}. These
expressions are similar to control expressions provided in programming
languages such as Scala or Java. A detailed description of the
expression language is beyond the scope of this paper.

\Klang{} also provides {\em multiplicities} as part of the
language. Multiplicities in \Klang{} are influenced by similar
concepts in languages such as UML/SysML. In \Klang{},
multiplicities can be used as a short hand for specifying collections
and also restricting the size of collections. Figure~\ref{fig:mult}
shows a \Klang{} model of a \code{Person} that has various member
properties, and the corresponding inferred type for each member
property. We will analyze each of these individually.

\begin{figure*}
  \centering
  \begin{tabular}[c]{c|c}
    \begin{subfigure}[c]{0.5\textwidth}
      \centering
      \begin{tabular}{c}
        \small
        \lstinputlisting{examples/mult.k}
      \end{tabular}
    \end{subfigure}
    \hspace{-0.5cm}
    &
    \begin{subfigure}[c]{0.5\textwidth}
      \centering
      \begin{tabular}{c}
        \small
        \lstinputlisting{examples/multr.k}
      \end{tabular}
    \end{subfigure}
    \\
  \end{tabular}
  \vspace{0.1cm}
  \caption{Example model (left) and inferred types (right) for members
    of class \code{Person}.}  
  \label{fig:mult}
\end{figure*}

Each \code{Person} can have exactly one \code{mother}. This is
specified by line 4. No explicit multiplicity is specified, which
makes it a singleton.  A \code{Person} can also have many unique
\code{children}, which is specified by line 5 using the \code{Set}
collection. Line 6 specifies that a \code{Person} may have many
\code{cars}. It is written using a modifier and a multiplicity, which
semantically translates to a \code{Set} (\Klang{} default for a
multiplicity is \code{Bag}) of \code{Car}. Finally, a person may own
one or more portfolios (\code{prtflios}, specified to have a
multiplicity of 1 or more), where each entry itself is a \code{Set} of
\code{Stock}. This translates to \code{prtflios} being a \code{Bag} of
\code{Set[Stock]} with at least 1 entry and no upper limit.

\sysml{} models can also carry meta data information in them
(sometimes introduced by tools). To accommodate for this, \Klang{}
also provides the {\em annotation} construct. New annotations can be
created and applied to classes, expressions, functions etc. Currently,
each annotation has a name and a type.

\Klang{} also provides single line comments using `-{}-' at the
beginning of the line, and block comments using `===(=*)' as the token for
the beginning and the end of the comment.

\subsection{K Type Checking}

The \Klang{} type checker performs basic checks on the provided input
to ensure naming and type consistency. It is used to ensure that all
declarations, expressions, annotations etc. are logically sound and
reference names (functions, members, variables) that exist and are
type consistent in the given context. Type information for all
expressions and any other inferences made by the type checker are
saved and made available to all other analyses/modules in the \Klang{}
tool chain. Further, the type checker imposes a stricter set of rules
on the provided input to ensure that it can be completely and
correctly translated to SMT. More details are provided in
Section~\ref{sec:k2smt}. The type checker is implemented as a stand
alone module, which is invoked after the AST has been constructed by a
visitor (interfacing with ANTLR). The implementation is done using
Scala.



\section{Change}
\label{sec:change}

\subsection{Behavior}

\Klang{} has to a large extent been used for specifying static structure, similar
to what can be represented by UML/SysML class diagrams, with requirements
typically constraining integer and real variables that represent properties of physical nature, such as for example weights and distances.
\Klang{} can, however, also be used for specifying behavior, using the same
concepts used for specifying structure, namely classes, properties, functions and
requirements on these. The idea is in other words to use mathematical logic to represent behaviors. This is an illustration of the pursued objective during 
the design of \Klang{} to keep the language as small as possible, relying as much as possible on the language of mathematics for expressing problems and solutions. 
This approach of course has algorithmic consequences when it comes to analyzing models. Our intention is to stay in mathematics as far as the tooling (existing theorem provers) allows us. In the following subsections we illustrate how one can encode two different behavior concepts in \Klang{}, namely state machines and event scheduling.

\subsubsection{State Machines}

State machines are commonly used to specify the behavior of software as well as hardware systems. They are frequently used at JPL for specifying the behavior of
embedded flight software software modules controlling, for example, planetary rovers. A state machine is defined by a collection of states,
a collection of events, and a labeled transition relation (labels are events) between states. This can of course be modeled in numerous ways. Here we shall assume deterministic state machines, and model the transition relation as a function. 
The \Klang{} model in Figure \ref{fig:statemachine} represents an encoding of a state machine modeling a rocket engine, which can be in one of the states:  \name{off}, \name{ready} or \name{firing}. Events include \name{turn\_on}, \name{fire}, and \name{turn\_off}. The types of states and events (\name{State} and \name{Event}) are modeled as body-less classes. The class \name{RocketEngine}
models our state machine. It defines the three states as well as the 
three events as properties of the appropriate types. A requirement expresses that the states are all different (a similar requirement should in principle also be provided for events).

The function \name{move} represents the transition relation, and 
is declared to take two arguments: a state and event, and to return a state. 
It has no body, meaning that it yet to be defined. The subsequent four  requirements define the 
\name{move} function. For example the first requirement states that
in the \name{off} state, on encountering a \name{turn\_on} event, the engine moves to the \name{ready} state. 

The last requirement demanding the engine to move from the \name{ready} state to the \name{off} state on a \name{fire} event was added a Monday morning by a tired
modeler, and in our context represents a change to the model.
This requirement, however, is inconsistent with a previous requirement 
that demands the resulting state to be \name{firing}.
Since \name{move} is a function (the transition relation is deterministic), and cannot return two
different values for the same argument, this is detected by the solver. 
Without this last requirement, the solver will declare the model satisfiable, and will synthesize the state machine function based on the provided requirements.
Note, however, that not all transitions are modeled, hence the synthesized state
machine may not be the desired one.

\notethis{KH: Wonder whether we need to name all requirements. Could just be simple names, such as r1, r2, ...}

\begin{figure}
\centering
\begin{tabular}{c}
\hline \\
\lstinputlisting{examples/statemachine.k} \\ \\
\hline
\end{tabular}
\caption{State machine}
\label{fig:statemachine}
\end{figure}

\subsubsection{Event Scheduling}

... bla bla ... Figure \ref{fig:scheduling} ...

\begin{figure}
\centering
\begin{tabular}{c}
\hline \\
\lstinputlisting{examples/scheduling.k} \\ \\
\hline
\end{tabular}
\caption{Scheduling}
\label{fig:scheduling}
\end{figure}

\subsection{Refinement}

Change can be considered as refinement. In the formal methods literature
refinement usually refers to the situation where one model/program, the specification, and typically abstract of nature, is replaced by a lower 
level model/program, the implementation. Along with the refinement normally
goes a proof, that the implementation refines the specification.
The literature offers many solutions to how specifications, implementations and refinements are expressed as well as proved correct, see for example \cite{vdm,raise}. We shall not here enumerate all of these, but bring forward two examples, one illustrating function refinement, and one illustrating data refinement.

\subsubsection{Function Refinement}

Function refinement consists of making the body of a function more concrete,
while the signature of the functions stays unchanged. More generally, data 
structures accessed by the function stay unchanged. One popular approach to this
is design by contract, where a function is first specified using pre/post conditions, and then later implemented with a function body. This form of refinement is advocated for example in specification languages such as 
\vdm{} \cite{vdm} and \raiselang{} \cite{raise}, as well as in programming
languages such as \eiffel{} \cite{eiffel} and \java{} in the form of the \jml{}
comment language \cite{jml}. 

\Klang{} supports design-by-contract using pre/post conditions. The example in
Figure \ref{fig:prepost} illustrates this with two class definitions. The left-most
class \name{Util\_Spec} represents the specification of a mathematics utility module containing two functions, \name{min} for computing the minimum of of
two values, and \name{abs} for returning the absolute value of a value. Both functions are specified with a post condition stating what is expected to be true
about the resulting value, denoted by \name{\$result}. The class \name{Util}
to the right extends class \name{Util\_Spec} and redefines the functions with
proper function bodies. The semantics of \Klang{} is such that the refined function bodies will have to satisfy the post conditions. The \Klang{} solver proves this
automatically in this case.

\begin{figure}
  \centering
  \begin{tabular}[c]{c|c}
    \hline \\
    Specification & Implementation \\
    \hline\hline \\ \\
    \begin{subfigure}[c]{0.5\textwidth}
     \lstinputlisting[lastline=13]{examples/prepost.k}
      \label{fig:prepost1}
    \end{subfigure}
    &
    \begin{subfigure}[c]{0.5\textwidth}
      \lstinputlisting[firstline=15,xleftmargin=3.0ex]{examples/prepost.k}
      \label{fig:prepost2}
    \end{subfigure}
    \\ \\
    \hline
  \end{tabular}    
  \caption{Mathematical function refinement}
  \label{fig:prepost}
\end{figure}

\subsubsection{Data Refinement}

\begin{figure}
  \centering
  \begin{tabular}[c]{c|c}
    \hline \\
    Specification & Implementation \\
    \hline\hline \\ \\
    \begin{subfigure}[c]{0.5\textwidth}
     \lstinputlisting[lastline=15]{examples/lightswitch.k}
      \label{fig:lightswitch1}
    \end{subfigure}
    &
    \begin{subfigure}[c]{0.5\textwidth}
      \lstinputlisting[firstline=17,lastline=26,xleftmargin=5.0ex]{examples/lightswitch.k}
      \label{fig:lightswicth2}
    \end{subfigure}
    \\ \\
    \hline
  \end{tabular}    
  \caption{Lightswitch refinement}
  \label{fig:lightswitch}
\end{figure}

\begin{figure}
\centering
\begin{tabular}{c}
\hline \\
\lstinputlisting[firstline=28]{examples/lightswitch.k} \\ \\
\hline
\end{tabular}
\caption{Lightswitch refinement proof}
\label{fig:lightswitch3}
\end{figure}


\section{CONCLUSION}
\label{sec:conclusion}

We have presented an overview of the \Klang{} language in this
paper. \Klang{} is intended to be used in a modeling environment for
proving satisfiability of \sysml{} models and exploring solutions to
various types of specifications, such as structure,
planning/scheduling, etc. We have also presented in detail, our
methodology for performing automatic translation of \Klang{}
specifications to SMT-LIB, and using an SMT solver such as \zthree{}
to perform model finding. Using manual methods of creating \Klang{}
specifications from \sysml{} models and reference materials, we have
already observed \Klang{} provide value in the modeling environment by
discovering unsatisfiability of scheduling problems in the proposed
Europa Clipper mission concept, which was confirmed by external manual
analysis.  In our current experience, \Klang{} seems to be sufficient
for creating small to medium sized \sysml{} models and proving
properties about them.
%
Concerning problems faced, a main challenge of course is the
higher-order nature of K, requested by mission engineers
(expressiveness prioritized over guaranteed analyzability).  SMT-LIB
is generally first-order.  Some problems are a consequence of using
SMT-LIB solvers, which struggle with the combination of arrays (used
for the heap and for sets) and universal quantification. Additionally,
the use of Real numbers and arithmetic on them is also a known SMT
challenge, especially in the context of arrays.
%
We are now in the process of creating tools to automatically translate
\sysml{} models to \Klang{} specifications (and back) and perform analysis on
them using the \Klang{} infrastructure. This will make it possible to view
a model as graphics as well as in text. The translation of K needs to
be extended to cover more constructs, including statements with
side-effects. Other challenges is making K executable by translation
to Scala, including executing OCL-like expressions,
and providing support for reflection such that models can query themselves.



\bibliographystyle{abbrv}
\bibliography{biblio}

\end{document}
