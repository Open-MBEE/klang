\section{Discussion}
\label{sec:discussion}

The work presented here follows a similar trend as the work related to
equivalence checking. \Klang{} provides a high level
programming/modeling language that can be used to express complex
ideas concisely. The primary purpose is to be able to express
constraints succinctly and be able to \emph{analyze} them in a useful
manner. \Klang's integration with SMT solvers provides us the ability
to do such an analysis and prove satisfiability and consistency of
class constraints. This is a natural precursor to dealing with
\emph{change}, which we propose and show can be dealt with by encoding
it as new or modified constraints. Often, modelers and programmers
tend to make changes without having any support for understanding the
change's impact at a global level. \Klang{} eases this task and
provides a natural method for creating and studying
changes. Additionally, in our experience, we have also observed that
modelers and programmers tend to use \Klang{} along with it's solving
ability as a tool for \emph{discovering} the right set of constraints
for their class before introducing a change. For example, uncertainty
about a particular variable and it's potential range of valid values
can be quite common in modeling environments. Since \Klang{} helps
discover unsatisfiability, we have observed modelers use an iterative
refinement technique to discover the appropriate range of a variable
for their needs. \Klang{} in this case is providing validation before
a change is completely committed.

