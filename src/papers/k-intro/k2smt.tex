
\section{Translating K to SMT-LIB}
\label{sec:k2smt}

In this section we illustrate the translation from K to the SMT- LIB
input language. SMT-LIB \cite{smt-lib} is the standard
``satisfiability modulo theories library'' for SMT solvers. The
standard is used by numerous SMT solvers, allowing comparison between
systems (for example in competitions).  In addition, it allows systems
generating SMT-LIB formulas to target any SMT solver processing this
standard. In our case we use the Z3 SMT solver \cite{de2008z3} to
process the generated formulas, but anticipate to target other solvers
as well in near term work.

\subsection{The Source K Model}

\begin{figure}
\centering
\begin{tabular}{c}
\hline \\
\lstinputlisting{examples/spacecraft.k} \\ \\
\hline
\end{tabular}
\caption{A simple \Klang{} model of a spacecraft}
\label{fig:spacecraftSmt}
\end{figure}

The translator currently covers a subset of the K language
corresponding to the model shown in Figure
\ref{fig:spacecraftSmt}. The example illustrates the features of K
that have been used by engineers at JPL until the time of writing. The
emphasis of these models is on {\em structure} of artifacts and {\em
  scheduling} of events. The model models a spacecraft. The class
\code{Object} is meant to represent entities that have
weight. Instruments, and its radio sub-classes, as well as the the
spacecraft itself, are objects, inheriting from this class. The class
\code{Instrument} defines a \code{power} level consumed. Requirements
in the form of Boolean constraints are imposed on \code{power} and
\code{weight}. The \code{SpaceCraft} class makes instances of
instruments, defines a combined sum \code{instrumentsWeight} and a
constraint on it with additional requirements. The elements of the
model discussed above are so-called {\em structural} elements, what
one would normally see in a SysML class diagram.

The spacecraft in addition contains a system manager, representing the
software on board. For the purpose of illustration, the system manager
is defined as a small {\em scheduler} of three events: a \code{bootUp}
event, re-booting the flight software computer, a \code{initMem}
event, initializing the computer memory, and a \code{takePicture}
event, taking a picture. An event is a constant of the \code{Event}
class, which defines an event as having a start time and an end time
appearing after the start time. In addition, the \code{Event} class
defines a function \code{after}, which as argument takes another event
`\code{e}', and returns true if the event (this) occurs after
`\code{e}'.  The \code{after} function is part of Allen logic
\cite{allen-logic-84} used by the planning and scheduling community.
%
Finally, the model contains an instance \code{ShRaan} of type
\code{SpaceCraft}.

Given the spacecraft model, the general proof-theoretic problem we
want an answer to is whether our classes are logically
consistent. That is, whether the constraints of each class are
consistent (do not evaluate to {\em false} such as for example is the
case with: `$x < 0 \wedge x > 0$'). From a semantics point of view, it
means that for each class there exists at least one instance (object)
of that class that satisfies the constraints.  The specific
satisfiability problem that perhaps interests a user most is whether
there is an instance \code{ShRaan} of the \code{SpaceCraft} class,
which satisfies all the constraints of that class and the classes it
refers to.

\subsection{The Translation to SMT-LIB}

The SMT-LIB input language (from here on referred to as SMT-LIB) is a
textual language for typed first order predicate logic plus various
theories, including for example arithmetic, unintepreted functions,
and arrays. The syntax is LISP-like, meaning for example that function
calls such as $f(42,false)$ have the form $(f\ 42\ false)$. For the 44
line K model in Figure \ref{fig:spacecraftSmt}, the translator
generates 333 lines of uncommented SMT-LIB code (additional comments
are generated to make the output easier for humans to read). We shall
below highlight subsets from each class of formulas generated, while
covering all the categories of formulas generated. Our main challenge
in translating K to SMT-LIB is how to translate classes 
supporting (multiple) {\em inheritance} and {\em recursive} references
between classes. This will be illustrated in the following.

\subsubsection{Classes, objects, and the heap}

Let's first translate a simple class, such as class \code{Object}.  We
have chosen to translate classes to the SMT-LIB concept of {\em
  datatypes}. A datatype in SMT-LIB corresponds to the classical
notion of an algebraic datatype: a named record, with a constructor
function that when applied to a sequence of values generates a value
of the datatype, while the values can be retrieved using
selector functions.  The class \code{Object} can be represented in SMT-LIB as
follows.

\lstset{language=SMT,numbers=none}

\begin{center}
\begin{tabular}{c}
\begin{lstlisting}
(declare-datatypes () ((Object 
  (mk-Object (weight Int)))))
\end{lstlisting}
\end{tabular}
\end{center}

This declaration declares the datatype \code{Object}, the constructor
\code{mk-Object}, which can be called on a value \code{w} of type
\code{Int} as follows: \code{(mk-Object w)}, to produce a value in
type \code{Object}. Reversely, given a value \code{o} in type
\code{Object}, we can retrieve the weight by applying the selector
function \code{weight} to \code{o} as follows: \code{(weight o)}.

Consider now the following schematic example of two mutually recursive
classes, a situation often occurring in SysML modeling (relationships
between two classes) as well as in programming (i.e. linked lists).

\lstset{language=K}

\begin{center}
\begin{tabular}{c}
\begin{lstlisting}
class A {
  b : B
}

class B {
  a : A
}
\end{lstlisting}
\end{tabular}
\end{center}

The following translation of this model to the SMT-LIB datatypes
\code{A} and \code{B} is {\bf not} well-founded since it contains a
recursion between \code{A} and \code{B} (it is illegal SMT-LIB).

\lstset{language=SMT,numbers=none}

\begin{center}
\begin{tabular}{c}
\begin{lstlisting}
(declare-datatypes () (
  (A (mk-A (b B)))
  (B (mk-B (a A)))
))
\end{lstlisting}
\end{tabular}
\end{center}

The solution is to operate with references to objects
rather than objects directly, exactly as done in a runtime system for an
object-oriented programming language. In other words, we need a
{\em heap} mapping references to objects. For this purpose must define
the type of references, which are just integers.

\begin{center}
\begin{tabular}{c}
\begin{lstlisting}
(define-sort Ref () Int)
\end{lstlisting}
\end{tabular}
\end{center}

We can now in SMT-LIB use \code{Ref} as the type of properties whos 
type in K is a class, they will now denote references to objects of the class.
This is illustrated by the following definition of the \code{SpaceCraft} datatype.

\begin{center}
\begin{tabular}{c}
\begin{lstlisting}  
(declare-datatypes () ((SpaceCraft 
  (mk-SpaceCraft (weight Int)
                 (instrumentsWeight Real)
                 (radio Ref)
                 (camera Ref)
                 (software Ref)))))
\end{lstlisting}
\end{tabular}
\end{center}

Observe how the fact that \code{SpaceCraft} inherits from
\code{Object} is modeled by the inclusion of the \code{weight} field
from \code{Object}. Inheritance is simply modeled by property
inclusion in this manner.  In order to define a heap, we need a
datatype that represents all the objects that can possibly be stored
in the heap. The following datatype \code{Any} represents all the
datatypes for the individual classes, by lifting them to this single
type (\code{null} is a zero argument constructor). The type \code{Any}
corresponds to Java's type \code{Object}.

\begin{center}
\begin{tabular}{c}
\begin{lstlisting}
(declare-datatypes () ((Any
  (lift-Object (sel-Object Object))
  (lift-Instrument (sel-Instrument Instrument))
  (lift-SpaceCraft (sel-SpaceCraft SpaceCraft))
  ...
  null))
)
\end{lstlisting}
\end{tabular}
\end{center}

Now we can define the heap as a mapping (an array) from references of
type \code{Ref} to \code{Any}.

\begin{center}
\begin{tabular}{c}
\begin{lstlisting}
(declare-const heap (Array Ref Any))
\end{lstlisting}
\end{tabular}
\end{center}

\subsubsection{Accessing the heap}

We first define a function \code{deref}, which when applied to a
reference returns the \code{Any} object at that entry.

\begin{center}
\begin{tabular}{c}
\begin{lstlisting}
(define-fun deref ((ref Ref)) Any
  (select heap ref)
)
\end{lstlisting}
\end{tabular}
\end{center}

With this function we are now ready to define functions, which can test
what kind of object is at a certain location in the heap, as well as
retrieve that object. The following functions perform these two tasks
for the case of the \code{Instrument} objects (for each datatype
constructor \code{C}, SMT-LIB generates a \code{is-C} function that
can determine whether an object is constructed with the constructor).

\begin{center}
\begin{tabular}{c}
\begin{lstlisting}
(define-fun deref-is-Instrument ((this Ref)) Bool
  (is-lift-Instrument (deref this))
)

(define-fun deref-Instrument ((this Ref)) Instrument
  (sel-Instrument (deref this))
)
\end{lstlisting}
\end{tabular}
\end{center}

As we have seen, K classes can contain properties of types that are
classes. For example the \code{SpaceCraft} class contains a property
\code{radio} of type \code{Instrument}. In an object-oriented language
like K with inheritance, such a property can denote any object that is
of type that either is equal to, or sub-classes \code{Instrument}. In
order to formulate invariants on objects of class \code{SpaceCraft},
we therefore need to be able to determine whether a \code{radio} object is equal
to, or sub-classes \code{Instrument}. This task is performed by the
following function, the body of which is a disjunction between the
three alternatives.

\begin{center}
\begin{tabular}{c}
\begin{lstlisting}
(define-fun deref-isa-Instrument ((this Ref)) Bool
  (or
    (deref-is-Instrument this)
    (deref-is-SimpleRadio this)
    (deref-is-SmartRadio this)
  )
)
\end{lstlisting}
\end{tabular}
\end{center}

\subsubsection{Getters of properties in classes}

Functions and requirements access properties. An example is the
expression \code{weight > 0} in class \code{Instrument}.  These
accesses are wrapped into {\em getter} functions. As an example, the
\code{weight} property of the class \code{Instrument} can be accessed
with a call of the following function, named \code{Instrument!weight}
(SMT-LIB allows symbols such as `\code{!}'  in names, to be discussed
further below), on a reference that is assumed to refer to an
\code{Instrument} object.

\begin{center}
\begin{tabular}{c}
\begin{lstlisting}
(define-fun Instrument!weight ((this Ref)) Int
  (weight (deref-Instrument this))
)
\end{lstlisting}
\end{tabular}
\end{center}

The above definition assumes that the \code{this} reference denotes an
\code{Instrument} object, and not an object of any sub-class on
\code{Instrument}, hence the `\code{!}' symbol (for {\em exact!
  class}) in the name.  This is sufficient when checking
satisfiability of the class \code{Instrument} class itself. However,
when checking the satisfiability of, for example, the \code{SpaceCraft}
class, which {\em contains} a property of type \code{Instrument}, as
for example \code{radio : Instrument}, we have to assume that
\code{radio} in addition potentially can refer to any object of a
class that sub-classes \code{Instrument}, which in this case is either
\code{SimpleRadio} or \code{SmartRadio}. This is achieved with the
following alternative getter function, named \code{Instrument.weight},
for the \code{weight} property of the class \code{Instrument}.

\begin{center}
\begin{tabular}{c}
\begin{lstlisting}
(define-fun Instrument.weight ((this Ref)) Int
  (ite (deref-is-Instrument this)         ; if
    (weight (deref-Instrument this))      ; then
    (ite (deref-is-SimpleRadio this)      ; else if
      (weight (deref-SimpleRadio this))   ; then
      (weight (deref-SmartRadio this))))) ; else
\end{lstlisting}
\end{tabular}
\end{center}

Each line in the body ends with a comment after the comment symbol
`\code{;}' explaining the structure of the LISP version of `${\bf
  if}\ e_1\ {\bf then}\ e_2\ {\bf else}\ e_3$', which is `$({\bf
  ite}\ e_1\ e_2\ e_3)$'. The reason for not just using the latter
more general function \code{Instrument.weight} for all accesses to the
\code{weight} property is that conditionals make it harder for an SMT
solver. Even moderately sized expressions with several accesses to
variables become unsolvable in reasonable time in the presence of such
conditional expressions. 


\subsubsection{Functions}

Functions are translated directly to SMT-LIB functions.  Each
function is translated in two versions, corresponding to the two
versions of the getter functions, and named using respectively
\code{className!functionName} and \code{className.functionName}, to
suggest which getter functions are called inside the function, again
depending on the calling context (whether \code{this} refers to the exact
class or potentially a sub-class). As an example, the following is the
translation of the \code{after} function in the class \code{Event}, only
showing one of the two versions, which are the same in this case.

\begin{center}
\begin{tabular}{c}
\begin{lstlisting}
(define-fun Event.after ((this Ref)(e Ref)) Bool
  (>= (Event.start this)  (Event.end e))
)
\end{lstlisting}
\end{tabular}
\end{center}

The first parameter is a reference (named \code{this}) of type
\code{Ref}. The \code{this} reference is meant to refer to the object
upon which the function is called. Consider for example a call like:
\code{takePicture.after(initMem)} in line 29 of Figure
\ref{fig:spacecraftSmt}. Here \code{takePicture} denotes a reference
to which the parameter \code{this} is bound.  The second parameter is
the user-provided parameter.

\subsubsection{Invariants and assertions}

We are finally able to present how class invariants are generated and
asserted. These validate the satisfiability of our classes.  The
invariant for a class is generated as a function that as argument
takes a \code{this} reference to an object of that class. Let's take
the example of the \code{SystemManager} class. The generated invariant
is the following.

\begin{center}
\begin{tabular}{c}
\begin{lstlisting}
(define-fun SystemManager.inv ((this Ref)) Bool
  (and
    (deref-isa-Event (SystemManager!bootUp this))
    (deref-isa-Event (SystemManager!initMem this))
    (deref-isa-Event (SystemManager!takePicture this))
    (and 
      (Event.after 
        (SystemManager!takePicture this)  
        (SystemManager!initMem this)) 
      (Event.after 
        (SystemManager!takePicture this)  
        (SystemManager!bootUp this))
    )
  )
)
\end{lstlisting}
\end{tabular}
\end{center}

The body of this function is a conjunction of the conditions that have
to hold on the \code{SystemManager} object referred to by
\code{this}. There are four such, three for the property definitions
in lines 26-28 in Figure \ref{fig:spacecraftSmt}, and one for the
requirement in line 29. Each of the property definitions results in a
condition that verifies that the property is of the right type, in
these three cases: that each of the properties \code{bootUp}, \code{initMem}, 
and \code{takePicture}, are objects of any sub-class of class \code{Event} (the use
of `\code{isa}'), although in this case there are no sub-classes of
\code{Event}. The last condition, corresponding to the requirement,
illustrate how functions are called, in the case the function
\code{after}.

We are now finally ready to assert the well-formedness of the
model. For each class two assertions are generated, one that asserts
the existence of an object of the class in the heap, and one asserting
that every object of that class in the heap satisfies the invariant of
that class. Below are these two assertions for the \code{SystemManager}
class.

\begin{center}
\begin{tabular}{c}
\begin{lstlisting}
(assert (exists ((this Ref)) 
  (deref-is-SystemManager this)))

(assert (forall ((this Ref))
  (=> 
    (deref-is-SystemManager this) 
    (SystemManager.inv this)
  )
))
\end{lstlisting}
\end{tabular}
\end{center}

\subsubsection{Solving the model}

Given the generated SMT-LIB model outlined above, an SMT solver
following the SMT-LIB standard can determine whether
the model is satisfiable. Our currently used SMT solver is Z3. If the
model is {\bf not} satisfiable, the solver will just return `not
satisfied', hence no information indicating why the model is not
satisfiable. One can in this case analyze subsets of the model,
eliminating assertions to discover which assertions caused the model
to become unsatisfiable, in the best case the minimal set of such
assertions. We are working on such a violation explanation capability.

\begin{figure}
\VerbatimInput{examples/spacecraftOutput.k}
\caption{Output of the K toolchain for the spacecraft example.}
\label{fig:shapes}
\end{figure}

If the model on the other hand is satisfiable, an assignment to
variables in the model will be retuned by the solver. In our case the
model outlined above is satisfiable. The returned assignment is shown
in Figure \ref{fig:shapes}. This view has been produced by processing
the output from Z3, which is less comprehensible.
%
The assignment shows the following. The outermost \code{ShRaan}
property in the heap denotes a \code{SpaceCraft} object.  This object
contains various fields, for example the \code{weight} property with
the value $18$, and the \code{software} property, which denotes the
reference (of type \code{Ref}) $21$. This reference in turn denotes a
\code{SystemManager} object containing three references \code{bootUp}
($25$), \code{initMem} ($26$), and \code{takePicture} ($27$), each of
which are events. Due to the constraint in line 29 of Figure
\ref{fig:spacecraftSmt} these events have been {\em scheduled} such
that the taking of the picture occurs after the boot as well as after
the memory initialization.  This can be seen from the fact that the
end times of the boot and memory initialization events at references
$25$ and $26$ are less than the start time of the take picture event
at reference $27$.

\lstset{language=K}
