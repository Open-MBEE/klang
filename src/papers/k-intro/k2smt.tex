\section{TRANSLATING \Klang{} to SMT-LIB}
\label{sec:k2smt}

In this section we illustrate the translation from K to the SMT- LIB
input language. SMT-LIB \cite{smt-lib} is the standard
``satisfiability modulo theories library'' for SMT solvers. The
standard is used by numerous SMT solvers, allowing comparison between
systems (for example in competitions).  In addition, it allows systems
generating SMT-LIB formulas to target any SMT solver processing this
standard. In our case we use the Z3 SMT solver \cite{de2008z3} to
process the generated formulas, but anticipate targeting other solvers
in the near term.

\subsection{The Source K Model}

\begin{figure}
\centering
\begin{tabular}{c}
%\hline \\
\small
\lstinputlisting{examples/spacecraft.k} \\ \\
%\hline
\end{tabular}
\caption{A simple \Klang{} model of a spacecraft}
\label{fig:spacecraftSmt}
\end{figure}

The translator currently covers a first-order logic subset of the K
language, corresponding to the model of a spacecraft shown in Figure
\ref{fig:spacecraftSmt}.
%
The translated subset includes classes, multiple inheritance,
properties of primitive types (\code{Bool}, \code{Int}, and
\code{Real}), user-defined class types, tuple types (cartesian
product), functions, pre/post conditions, and constraints.  Functions,
pre/post conditions, and constraints can be specified in a rich
expression language supporting conditionals, class constructors, dot
notation for accessing properties in objects, and universal and
existential quantification.  Sets are under development but not
covered here. Currently not translated constructs include type
parameterized classes, statements with side-effects (assignment) and
their corresponding looping constructs, functions as first class
citizens, type abbreviations and predicate subtypes, as well as
multiplicities, which will be treated as collection types.  Recursive
functions can currently not be defined using function definitions, but
can be specified by providing function signatures plus separate
constraints.

The example illustrates the features of \Klang{} that have been used
by engineers at JPL until the time of writing. The emphasis of these
models is on {\em structure} of artifacts and {\em scheduling} of
events. The class \code{Thing} is meant to represent entities that
have weight. Instruments, its radio sub-classes, and the
\code{SpaceCraft} class itself, inherit from class \code{Thing}. Class
\code{Instr} defines a \code{power} level. Requirements in the form of
Boolean constraints are imposed on \code{power} and \code{weight}. The
\code{SpaceCraft} class makes instances of instruments, defines a
combined sum \code{instrWeight}, and a constraint on it with
additional requirements. Such elements of a model are so-called {\em
  structural} elements, what one would normally see in a SysML class
diagram.

The spacecraft in addition contains a system manager, representing the
software on board. For the purpose of illustration, the system manager
is defined as a small {\em scheduler} of three events: a \code{bootUp}
event, re-booting the flight software computer, a \code{initMem}
event, initializing the computer memory, and a \code{tkPic}
event, taking a picture. An event is an instance of the \code{Event}
class, which defines an event as having a start time and an end time
appearing after the start time. In addition, the \code{Event} class
defines a function \code{after}, which as argument takes another event
`\code{e}', and returns true if the event (\code{this}) occurs after
`\code{e}'.  The definition of the \code{after} function is inspired by 
Allen logic \cite{allen-logic-84}.
%
Finally, the model contains an instance \code{ShRaan} of type
\code{SpaceCraft}.

Given the spacecraft model, the general proof-theoretic problem we
want an answer to is whether our classes are logically
consistent. That is, whether the constraints of each class are
consistent (do not evaluate to {\em false} such as for example is the
case with: `$x < 0 \wedge x > 0$'). From a semantics point of view, it
means that for each class there exists at least one instance (object)
of that class that satisfies the constraints.  An SMT solver demonstrates
satisfiability by finding a model satisfying the constraints
(model finding). This model furthermore represents a schedule of the events in
the system manager.

%The specific
%satisfiability problem that perhaps interests a user most is whether
%there is an instance \code{ShRaan} of the \code{SpaceCraft} class,
%which satisfies all the constraints of that class and the classes 
%to which it refers.

\subsection{The Translation to SMT-LIB}

The SMT-LIB language (from here on referred to as SMT-LIB) is a textual
language for typed first order predicate logic plus various theories,
including for example arithmetic, uninterpreted functions, and
arrays. The syntax is LISP-like, meaning for example that function
calls such as $f(42,false)$ have the form $(f\ 42\ false)$. For the 44
line \Klang{} model in Figure \ref{fig:spacecraftSmt}, the translator
generates 333 lines of uncommented SMT-LIB code (additional comments
are generated to make the output easier for humans to read). We shall
below highlight subsets from each class of formulas generated, while
covering each of the categories of formulas generated. Our main
challenge in translating K to SMT-LIB is how to translate classes
supporting (multiple) {\em inheritance} and {\em recursive} references
between classes. This will be illustrated in the following.

\textbf{Classes, objects, and the heap} Let's first translate a simple
class, such as class \code{Thing}.  We have chosen to translate
classes to the SMT-LIB concept of {\em datatypes}. A datatype in
SMT-LIB corresponds to the classical notion of an algebraic datatype:
a named record, with a constructor function that when applied to a
sequence of values generates a value of the datatype, while the values
can be retrieved using selector functions.  The class \code{Thing}
can be represented in SMT-LIB as follows.

\lstset{language=SMT,numbers=none}

\begin{center}
\begin{tabular}{c}
\small
\begin{lstlisting}
(declare-datatypes () ((Thing 
  (mk-Thing (weight Int)))))
\end{lstlisting}
\end{tabular}
\end{center}

\noindent This declaration declares the datatype \code{Thing}, the
constructor \code{mk-Thing}, which can be called on a value \code{w}
of type \code{Int} as follows: \code{(mk-Thing w)}, to produce a
value in type \code{Thing}. Reversely, given a value \code{o} in type
\code{Thing}, we can retrieve the weight by applying the selector
function \code{weight} to \code{o} as follows: \code{(weight o)}.

Consider now the following schematic example of two mutually recursive
classes, a situation often occurring in SysML modeling (relationships
between two classes) as well as in programming (i.e. linked lists).
Note that due to the declarative nature of K, it is possible to initialize
objects of such classes in a recursive manner even without programming with side-effects.
If no constructors are applied, the solver will assign objects.

\lstset{language=K,numbers=none}

\begin{center}
\begin{tabular}{c}
\small
\begin{lstlisting}
class A {   class B {
  b : B          a : A
}           }
\end{lstlisting}
\end{tabular}
\end{center}

\noindent The following translation of this model to the SMT-LIB
datatypes \code{A} and \code{B} is {\bf not} well-founded since it
contains recursion between \code{A} and \code{B} (it is illegal
SMT-LIB).

\lstset{language=SMT,numbers=none}

\begin{center}
\begin{tabular}{c}
\small
\begin{lstlisting}
(declare-datatypes () (
  (A (mk-A (b B)))
  (B (mk-B (a A)))))
\end{lstlisting}
\end{tabular}
\end{center}

\noindent The solution is to operate with {\em references} to objects rather
than objects directly, exactly as done in any runtime system for an
object-oriented programming language. In other words, we need a {\em
  heap} mapping references to objects. For this purpose we define
the type of references as integers.

\begin{center}
\begin{tabular}{c}
\small
\begin{lstlisting}
(define-sort Ref () Int)
\end{lstlisting}
\end{tabular}
\end{center}

\noindent We can now in SMT-LIB use \code{Ref} as the type of properties whose 
type in K is a class, they will now denote references to objects of the class.
This is illustrated by the following definition of the \code{SpaceCraft} datatype.

\begin{center}
\begin{tabular}{c}
\small
\begin{lstlisting}  
(declare-datatypes () ((SpaceCraft 
  (mk-SpaceCraft 
      (weight Int)
      (instrWeight Real)
      (radio Ref) (camera Ref)
      (software Ref)))))
\end{lstlisting}
\end{tabular}
\end{center}

\noindent Observe how the fact that \code{SpaceCraft} inherits from
\code{Thing} is modeled by the inclusion of the \code{weight} field
from \code{Thing}. Inheritance is simply modeled by property
inclusion in this manner.  In order to define a heap, we need a
datatype that represents all the objects that can possibly be stored
in the heap. The following datatype \code{Any} represents all the
datatypes for the individual classes, by lifting them to this single
type (\code{null} is a zero argument constructor). The type \code{Any}
corresponds to Java's type \code{Object}.

\begin{center}
\begin{tabular}{c}
\small
\begin{lstlisting}
(declare-datatypes () ((Any
  (lift-Thing 
    (sel-Thing Thing))
  (lift-Instr 
    (sel-Instr Instr))
  (lift-SpaceCraft 
    (sel-SpaceCraft SpaceCraft))
  ...
  null)))
\end{lstlisting}
\end{tabular}
\end{center}

\noindent Now we can define the heap as an array from references of
type \code{Ref} to \code{Any}.

\begin{center}
\begin{tabular}{c}
\small
\begin{lstlisting}
(declare-const heap (Array Ref Any))
\end{lstlisting}
\end{tabular}
\end{center}

\textbf{Accessing the heap} We first define a function \code{deref},
which when applied to a reference returns the \code{Any} object at
that entry.

\begin{center}
\begin{tabular}{c}
\small
\begin{lstlisting}
(define-fun deref ((ref Ref)) Any
  (select heap ref))
\end{lstlisting}
\end{tabular}
\end{center}

\noindent With this function we are now ready to define functions,
which can test what kind of object is at a certain location in the
heap, as well as retrieve that object. The following functions perform
these two tasks for the case of the \code{Instr} objects (for
each datatype constructor \code{C}, SMT-LIB generates an \code{is-C}
function that can determine whether an object is constructed with the
constructor).

\begin{center}
\begin{tabular}{c}
\small
\begin{lstlisting}
(define-fun deref-is-Instr 
  ((this Ref)) Bool
  (is-lift-Instr (deref this)))

(define-fun deref-Instr 
  ((this Ref)) Instr
  (sel-Instr (deref this)))
\end{lstlisting}
\end{tabular}
\end{center}

\noindent As we have seen, K classes can contain properties of types
that are classes. For example the \code{SpaceCraft} class contains a
property \code{radio} of type \code{Instr}. In an object-oriented
language like K with inheritance, such a property can denote any
object that is of type that either is equal to, or sub-classes
\code{Instr}. In order to formulate invariants on objects of
class \code{SpaceCraft}, we therefore need to be able to determine
whether a \code{radio} object is equal to, or sub-classes
\code{Instr}. This task is performed by the following function,
the body of which is a disjunction between the three alternatives.

\begin{center}
\begin{tabular}{c}
\small
\begin{lstlisting}
(define-fun deref-isa-Instr 
  ((this Ref)) Bool
  (or
    (deref-is-Instr this)
    (deref-is-SmplRadio this)
    (deref-is-SmrtRadio this)))
\end{lstlisting}
\end{tabular}
\end{center}

\textbf{Getters of properties in classes} Functions and requirements
access properties. An example is the expression \code{weight > 0} in
class \code{Instr}.  These accesses are wrapped into {\em getter}
functions. As an example, the \code{weight} property of the class
\code{Instr} can be accessed with a call of the following
function, named \code{Instr!weight} (SMT-LIB allows symbols such
as `\code{!}'  in names, to be discussed further below), on a
reference that is assumed to refer to an \code{Instr} object.

\begin{center}
\begin{tabular}{c}
\small
\begin{lstlisting}
(define-fun Instr!weight 
  ((this Ref)) Int
  (weight (deref-Instr this)))
\end{lstlisting}
\end{tabular}
\end{center}

\noindent The above definition assumes that the \code{this} reference
denotes an \code{Instr} object, and not an object of any
sub-class on \code{Instr}, hence the `\code{!}' symbol (for {\em
  exact!  class}) in the name.  This is sufficient when checking
satisfiability of the class \code{Instr} class itself. However,
when checking the satisfiability of, for example, the
\code{SpaceCraft} class, which {\em contains} a property of type
\code{Instr}, as for example \code{radio : Instr}, we have
to assume that \code{radio} in addition potentially can refer to any
object of a class that sub-classes \code{Instr}, which in this
case is either \code{SmplRadio} or \code{SmrtRadio}. This is
achieved with the following alternative getter function, named
\code{Instr.weight}, for the \code{weight} property of the class
\code{Instr}.

\begin{center}
\small
\begin{lstlisting}
(define-fun Instr.weight 
  ((this Ref)) Int
  ; if
  (ite (deref-is-Instr this)         
    ; then
    (weight (deref-Instr this))      
    ; else if
    (ite (deref-is-SmplRadio this)      
      ; then
      (weight (deref-SmplRadio this))   
      ; else
      (weight (deref-SmrtRadio this))
    )))
\end{lstlisting}
\end{center}

\noindent Each line in the body ends with a comment after the comment
symbol `\code{;}' explaining the structure of the LISP version of
`${\bf if}\ e_1\ {\bf then}\ e_2\ {\bf else}\ e_3$', which is `$({\bf
  ite}\ e_1\ e_2\ e_3)$'. The reason for not just using the latter
more general function \code{Instrument.weight} for all accesses to the
\code{weight} property is that conditionals make it harder for an SMT
solver. Even moderately sized expressions with several accesses to
variables become unsolvable in reasonable time in the presence of such
conditional expressions.

\textbf{Functions} Functions are translated directly to SMT-LIB
functions.  Each function is translated in two versions, corresponding
to the two versions of the getter functions, and named using
respectively \code{className!functionName} and
\code{className.functionName}, to suggest which getter functions are
called inside the function, again depending on the calling context
(whether \code{this} refers to the exact class or potentially a
sub-class). As an example, the following is the translation of the
\code{after} function in the class \code{Event}, only showing one of
the two versions, which are the same in this case.

\begin{center}
\begin{tabular}{c}
\small
\begin{lstlisting}
(define-fun Event.after 
  ((this Ref)(e Ref)) Bool
  (>= (Event.start this)  (Event.end e)))
\end{lstlisting}
\end{tabular}
\end{center}

\noindent The first parameter is a reference (named \code{this}) of
type \code{Ref}. The \code{this} reference is meant to refer to the
object upon which the function is called. Consider for example a call
like: \code{tkPic.after(initMem)} in line 31 of Figure
\ref{fig:spacecraftSmt}. Here \code{tkPic} denotes a reference
to which the parameter \code{this} is bound.  The second parameter is
the user-provided parameter.

%\subsection{Invariants and assertions}

\textbf{Invariants and assertions} We are finally able to present how
class invariants are generated and asserted. These validate the
satisfiability of our classes.  The invariant for a class is generated
as a function that as argument takes a \code{this} reference to an
object of that class. Let's take the example of the
\code{SysMngr} class. The generated invariant is the following.

\begin{center}
\begin{tabular}{c}
\small
\begin{lstlisting}
(define-fun SysMngr.inv 
  ((this Ref)) Bool
  (and
    (deref-isa-Event 
      (SysMngr!bootUp this))
    (deref-isa-Event 
      (SysMngr!initMem this))
    (deref-isa-Event 
      (SysMngr!tkPic this))
    (and 
      (Event.after 
        (SysMngr!tkPic this)  
        (SysMngr!initMem this)) 
      (Event.after 
        (SysMngr!tkPic this)  
        (SysMngr!bootUp this)))))
\end{lstlisting}
\end{tabular}
\end{center}

\noindent The body of this function is a conjunction of the conditions
that have to hold on the \code{SysMngr} object referred to by
\code{this}. There are four such: three for the property definitions
in lines 28 -- 30 in Figure \ref{fig:spacecraftSmt}, and one for the
requirement on line 31. Each of the property definitions results in a
condition that verifies that the property is of the right type, in
these three cases: that each of the properties \code{bootUp},
\code{initMem}, and \code{tkPic}, are objects of any sub-class of
class \code{Event} (the use of `\code{isa}'), although in this case
there are no sub-classes of \code{Event}. The last condition,
corresponding to the requirement, illustrate how functions are called,
in the case the function \code{after}.

We are now finally ready to assert the well-formedness of the
model. For each class two assertions are generated, one that asserts
the existence of an object of the class in the heap, and one asserting
that every object of that class in the heap satisfies the invariant of
that class. Below are these two assertions for the \code{SysMngr}
class.

\begin{center}
\begin{tabular}{c}
\small
\begin{lstlisting}
(assert (exists ((this Ref)) 
  (deref-is-SysMngr this)))

(assert (forall ((this Ref))
  (=> 
    (deref-is-SysMngr this) 
    (SysMngr.inv this))))
\end{lstlisting}
\end{tabular}
\end{center}

\textbf{Solving the model} Given the generated SMT-LIB model outlined
above, an SMT solver following the SMT-LIB standard can determine
whether the model is satisfiable. Our currently used SMT solver is
Z3. If the model is {\bf not} satisfiable, the solver will just return
`not satisfied'. One can in this case analyze subsets of the model,
eliminating assertions to discover which assertions caused the model
to become unsatisfiable, in the best case the minimal set of such
assertions. We are working on such a violation explanation capability.


\begin{figure*}
\centering
  \lstset{language=SMT,numbers=none}
  \small
  \begin{tabular}{c}
    \scalebox{0.9}{\lstinputlisting{examples/spacecraftOutput.k}}
  \end{tabular}
  \caption{Output of the K tool chain for the spacecraft example.}
  \label{fig:shapes}
\end{figure*}


If the model on the other hand is satisfiable, an assignment to
variables in the model will be returned by the solver. In our case the
model outlined above is satisfiable and solves in 2 seconds. 
The returned assignment is shown
in Figure \ref{fig:shapes}. This view has been produced by processing
the output from Z3, which is less comprehensible.
%
The assignment shows the following. The outermost \code{ShRaan}
property in the heap denotes a \code{SpaceCraft} object.  This object
contains various fields, for example the \code{weight} property with
the value $18$, and the \code{software} property, which denotes the
reference (of type \code{Ref}) $21$. This reference in turn denotes a
\code{SysMngr} object containing three references \code{bootUp}
($25$), \code{initMem} ($26$), and \code{tkPic} ($27$), each of
which are events. Due to the constraint in line 31 of Figure
\ref{fig:spacecraftSmt} these events have been {\em scheduled} such
that the taking of the picture occurs after the boot as well as after
the memory initialization.  This can be seen from the fact that the
end times of the boot and memory initialization events at references
$25$ and $26$ are less than the start time of the take picture event
at reference $27$. Note that the values suggested by the SMT solver 
are not necessarily realistic, although they satisfy the provided
constraints.

\lstset{language=K}
