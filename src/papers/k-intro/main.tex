\documentclass{llncs}

\title{K: A Wide Spectrum Language for Modeling, Programming, and
  Analysis\thanks{ The work described in this publication was carried
    out at Jet Propulsion Laboratory, California Institute of
    Technology, under a contract with the National Aeronautics and
    Space Administration.}}

\author{
  Klaus Havelund 
  \and 
  Rahul Kumar 
  \and 
  Chris Delp
  \and
  Bradley Clement
}

\institute{ Jet Propulsion Laboratory\\ 
            California Institute of Technology\\ 
            California, USA }

\usepackage{multirow}
\usepackage{multicol}
%\usepackage[T1]{fontenc}
%\usepackage[latin9]{inputenc}
%\usepackage{babel}
%\usepackage{alltt}
%\usepackage{comment}
%\usepackage{multirow}
%\usepackage{graphicx}

% -----------------
% --- comments: ---
% -----------------

\newcommand{\notethis}[1]{
  \color{red}
  \vspace{0.3cm}
  \noindent\makebox[\linewidth]{\rule{\paperwidth}{0.4pt}}\\
  {\sc NOTE - } #1\\
  \noindent\makebox[\linewidth]{\rule{\paperwidth}{0.4pt}}\\
  \vspace{0.3cm}
  \color{black}
}

% ------------------
% --- listings: ---
% ------------------

\usepackage{url}
\usepackage{listings}
\usepackage{color}
\usepackage{amssymb}
\usepackage{framed}
\usepackage{graphicx} 
\usepackage{caption} 
\captionsetup{compatibility=false}
\usepackage{subcaption}
\usepackage{fancyvrb}
\usepackage{verbatimbox}
\usepackage{verbatim}
\usepackage{fancyvrb}
\usepackage{apalike}
\usepackage{SCITEPRESS}

\definecolor{light-gray}{gray}{0.95}

\lstdefinelanguage{K}{
  morekeywords={package,import,class,unique,extends,with,match,new,if,then,else,while,for,yield,val,var,this,case,type,implicit,private,protected,abstract,final, req, Int, String, Bool, fun, pre, post},
  otherkeywords={&,\{,\},_*},
  literate=
      {\#\#}{{$\oplus\;$}}2
      {|->}{{$\longmapsto\;$}}2
      {-->}{{$\Longrightarrow\;$}}2
      {->}{{$\rightarrow\;$}}2
      {~}{{$\sim\;$}}2
      {>>}{{$\gg$}}2
      {++}{{$++$}}2
      {|==}{{$\models$}}2
  ,
  sensitive=true, 
  morecomment=[l]{//}, 
  morecomment=[s]{/*}{*/},
  stringstyle=\ttfamily,
  numbers=left,   
  numberstyle=\footnotesize, 
}

\lstdefinelanguage{SMT}{
  morekeywords={define,declare,sort,const,fun,datatypes,assert,forall,exists,and,or,ite,select,Int,Real,Bool,Array},
  sensitive=true, 
  morecomment=[l]{;}, 
  stringstyle=\ttfamily,
%  numbers=none,   
%  numberstyle=\footnotesize, 
}

 
% --- Referring to code: ---

\newcommand{\name}[1]{{\em #1}}
%\newcommand{\code}[1]{\texttt{#1}}
\newcommand{\issue}[2]{{\bf ** #1: #2}}

\newcommand{\Klang}{{K}}
\newcommand{\tracecontract}{{TraceContract}}
\newcommand{\logfire}{{LogFire}}
\newcommand{\vdm}{VDM}
\newcommand{\vdmpp}{\vdm$^{++}$}
\newcommand{\asml}{Asml}
\newcommand{\raiselang}{Raise}
\newcommand{\rsl}{RSL}
\newcommand{\zlang}{Z}
\newcommand{\alloy}{Alloy}
\newcommand{\clear}{Clear}
\newcommand{\eiffel}{Eiffel}
\newcommand{\dafny}{Dafny}
\newcommand{\whythree}{Why3}
\newcommand{\whyml}{WhyML}
\newcommand{\eml}{EML}
\newcommand{\specsharp}{SpeC$^{\#}$}
\newcommand{\java}{Java}
\newcommand{\clang}{C}
\newcommand{\scala}{Scala}
\newcommand{\sml}{SML}
\newcommand{\ml}{ML}
\newcommand{\haskell}{Haskell}
\newcommand{\fortress}{Fortress}
\newcommand{\python}{Python}
\newcommand{\ocaml}{Ocaml}
\newcommand{\jml}{JML}
\newcommand{\uml}{UML}
\newcommand{\sysml}{SysML}
\newcommand{\ltl}{LTL}
\newcommand{\pvs}{PVS}
\newcommand{\isabelle}{Isabelle}
\newcommand{\coq}{Coq}
\newcommand{\boogie}{Boogie}
\newcommand{\zthree}{Z3}
\newcommand{\ems}{EMS}
\newcommand{\antlr}{ANTLR}
\newcommand{\slam}{SLAM}
\newcommand{\spin}{SPIN}
\newcommand{\promela}{Promela}{

\newcommand*{\scaleFactor}{0.75}

% \lstset{language=K,numbers=left,numberstyle=\scriptsize,numbersep=-21pt}
\lstset{language=K}


\begin{document}

\maketitle


\begin{abstract}

The formal methods community has over the years proposed various 
formally founded specification languages based on predicate logic 
and set theory. At the same time the model-based engineering community 
has proposed less formally founded graphical formalisms such as UML and 
SysML. We report on an effort to formally ground SysML in a textual 
formal language, named K, supporting classes, multiple inheritance, predicate 
logic and set theory. K contains programming constructs, and can thus be 
considered as a  wide-spectrum modeling and programming language. We further 
explain the translation of a subset of this textual language to the input 
language of the SMT-LIB standard, and the application of Z3 for analysis 
of the generated SMT-LIB formulas. The entire effort is part of a larger 
effort to develop a general purpose SysML development framework for designing 
systems, currently supporting the design of NASA's planned 2022 mission to 
Jupiter's Moon Europa. 

\end{abstract}


\keywords{Modeling, programming, constraints, refinement,
  verification, SMT, analysis.}


\section{Introduction}
\label{sec:introduction}

Modeling is the activity of formulating an abstract description of a
system to be implemented (or possibly already implemented). Modeling
includes such activities as requirements capture in the initial phases
and design of higher-level architectural decisions in later
stages. Modeling has been studied by various communities, of which at
least two can be identified: the model-based engineering community and
the formal methods community. The {\em model-based engineering}
community has suggested graphical modeling languages such as UML
\cite{uml} and SysML \cite{sysml}, a variant of UML.  Both have been
designed by the OMG (Object Management Group) technology standards
consortium. SysML is meant for systems development more broadly
considered, including physical systems as well as software systems, in
contrast to UML, which is mainly meant for software development. These
graphical formalisms have received a high degree of popularity in
industry due to their two dimensional format, also sometimes referred
to popularly as {\em boxology}: boxes and arrows. However, drawbacks
of these formalisms include lack of precise semantics, lack of
analysis capabilities, tedious GUI operations, requiring lots
of visual real estate even for simple models, as well as large volumes
of technologies. Learning UML and SysML is not just learning very
large languages, it is also learning a large set of additional tools
needed to work with models. We formulate the hypothesis that some of these
drawbacks in part are due to the lack of a simple textual language, at
a size comparable to a programming language, underlying the
graphical notations.

From an even earlier point in time, since the 1960s, the {\em
  formal methods} community, part of the computer science community,
and closely connected to the programming language community, has
proposed numerous formally founded specification languages. Several of
these are based on predicate logic and set theory. These languages
are, compared to UML and SysML, concise, small, well defined in the
form of semantics, and in recent time well supported with analysis
capabilities. The obvious observation is that it might be fruitful to
study the interaction between the two classes of formalisms. Consider
furthermore that programming languages are gaining in abstraction,
such as for example combining object-oriented and functional
programming. An example is Scala, which has many commonalities with very early formal
specification languages, such as for example VDM \cite{vdm78}, and
specifically its object-oriented variant VDM$^{++}$
\cite{vdmplusplus05}.  The study should therefore include the
interaction between the graphical formalisms, the formal methods
modeling languages, and programming languages.

We report on an effort to formally ground SysML in a textual
formal specification language, named K (Kernel language), designed
specifically for this purpose.  
Our initial focus is on specifying and analyzing class diagrams
with constraints. K supports object-oriented concepts
such as classes, multiple inheritance, and object instances. The
contents of classes can be typed values, including functions, and
constraints over these expressed in higher-order predicate logic. 
K also contains programming constructs such as
variables, assignment statements, and looping constructs, and can as
such be seen as a wide-spectrum modeling and programming language.
The K language can be seen as a vehicle for giving semantics to SysML,
providing analysis capabilities, and even provide an alternative to
modeling with the mouse: writing textual models in K directly, just
like one normally writes programs.  We adhere to the school of thought
that modeling can be seen as programming in a language where some parts of the
model (program) at any point in time are executable, and some maybe
are not (yet).

The idea of merging modeling and programming in one language has been
suggested before, as will be discussed in the related work section.
Although K is not much different from previously suggested formalisms,  
our contribution is the creation of \Klang{} specifically in support of a
SysML model engineering tool set under development, to be used by
designers of the proposed 2022 mission to Jupiter's Moon Europa, also
referred to as the Europa Clipper mission concept
\cite{europa-clipper}.  The resulting tool set will support graphical
SysML modeling using MagicDraw \cite{magicdraw}, as well as
browser-based model viewing and editing, including of textual K
models. A first-order subset of K is furthermore translated to the input language of the
SMT-LIB standard, and currently processed with the Z3 SMT solver for
proving satisfiability of class definitions (are the constraints
consistent?), and model finding (find variable assignments satisfying
constraints, for example used in task scheduling).  The contribution here is
the handling of (multiple) class inheritance, which is typically not supported
by similar languages translated to SMT-LIB, as well as the allowance
of recursive class definitions. Multiple inheritance is a crucial part
of SysML, and therefore of K.

%The idea of merging modeling and programming in one language has been
%studied before, as will be discussed in the related work section. Our
%contribution is the development of \Klang{}, along with the technology
%to automatically translate it to SMT (for proving satisfiability) in
%the presence of multiple inheritance and constraints. This technology
%as part of a larger effort at NASA's Jet Propulsion Laboratory (JPL)
%to develop a SysML model engineering tool set to be used by designers
%of the proposed 2022 mission to Jupiter's Moon Europa, also referred to
%as the Europa Clipper mission concept \cite{europa-clipper}. This tool set,
%named EMS (Engineering Modeling System), supports graphical SysML
%modeling using MagicDraw \cite{magicdraw} as well as textual K
%modeling using any text editor of convenience. 

The paper is organized as follows. Section \ref{sec:k-syntax}
introduces a subset of the K language through an example, which is
similar to examples typically used to illustrate such formal
specification languages. Section \ref{sec:k2smt} outlines the
translation from K to the SMT-LIB input language for the purpose of
analysis of K models. This section is based on a different example
illustrating how K is actually currently used at JPL. Section
\ref{sec:usage} explains the integration of K within the SysML
development framework, as well as the usage of this. Section
\ref{sec:related-work} discusses related work, and finally Section
\ref{sec:conclusion} concludes the paper. Appendix \ref{app:grammar}
contains the grammar for K in ANTLR \cite{antlr} format.


\section{K Syntax}

Contains one example that we discuss in detail. 

\begin{figure}
\centering
\begin{tabular}{c}
\hline \\
\lstinputlisting{examples/fs.k} \\ \\
\hline
\end{tabular}
\caption{A simple model of a spacecraft using \Klang{}.}
\label{fig:spacecraftSmt}
\end{figure}

\begin{enumerate}
\item Classes
\item Inheritance
\item Functions
\item Constraints
\item undefined functions
\item Sequences and Sets
\item Constructor with named arguments
\item Primitive types
\item Expressions
\item Function specifications
\item Quantification
\item Bounded types
\end{enumerate}

\subsection{K Type Checking}

Will use the same example as presented for the syntax.

\section{K to SMT}

Will probably contain one large example, which is then discussed in
smaller individual bits and pieces, where each contains the K code and
the SMT code

\begin{figure}
\centering
\begin{tabular}{c}
\hline \\
\lstinputlisting{examples/spacecraft.k} \\ \\
\hline
\end{tabular}
\caption{A simple model of a spacecraft using \Klang{}.}
\label{fig:spacecraftSmt}
\end{figure}

\begin{figure}
\VerbatimInput{examples/spacecraftOutput.k}
\caption{Output of the K toolchain for the spacecraft example.}
\label{fig:shapes}
\end{figure}

\subsection{Limitations of SMT Translation}

\section{Usage Scenario In Practice}

The idea here is to tell the story of how people create models in
MagicDraw, show screenshot maybe, followed by how we can automatically
translate that to a K model and then do analysis on it. Show the
entire pipeline.

\section{Discussion}


\section{Related Work}
\label{sec:related-work}

\Klang{} is intended to represent a textual modeling language capable of representing \sysml{} concepts, specifically class diagrams with constraints.
However, as mentioned in the introduction, it also contains programming constructs.
As such it can be perceived as a wide-spectrum modeling/programming language.

Wide spectrum specification languages have been investigated to length in the formal 
methods community. One of the most well-known examples is \vdm{} 
\cite{vdm78,bjoerner-jones-82,jones90,jones-shaw-90}. \vdm{} in its
original form \cite{vdm78} provided a combination of procedural programming and
functional programming and specification using sets, lists and maps (with proper 
mathematical notation), and higher-order predicate logic. \vdmpp{} 
\cite{vdmplusplus05} added object-orientation to \vdm{}, which is now part of
the \vdm{} standard \cite{vdmsl}. The \raiselang{} specification language (\rsl{})
\cite{raise92} is a wide-spectrum language taking inspiration from \vdm{} as well as 
from other modeling languages such as \zlang{} and from algebraic specification 
languages, such as \clear{}. \asml{} \cite{asml05} is a more recent wide-spectrum
specification language, in many ways similar to \vdm{}, but based on the fundamental concept that operations operate on algebras.

Several high-level programming languages have been developed in recent years,
including  the early \sml{} (Standard ML) \cite{standard-ml-97}, its derivative
\ocaml{} \cite{ocaml}, and \haskell{} \cite{haskell}. However, also \java{}
can be considered high-level due to its libraries of collections (sets, lists, and 
maps), as well as the iterator concept. \python{} \cite{python} is close to 
combining object-oriented and functional programming. The 
\scala{} \cite{scala} language does this to the full extent, as does the
\fortress{} \cite{fortress}.

Specification constructs have been introduced in programming languages, in the form
of design-by-contract (pre/post conditions + class invariants). Examples are
\eiffel{} \cite{eiffel} and \specsharp{} \cite{specsharp}, where contracts are 
part of the language. \scala{} has library functions for writing pre/post conditions 
on functional programs \cite{odersky-rv10}. Finally, The \jml{} language \cite{jml} 
allows to write design-by-contract specifications for \java{} as comments. These are 
ignored by the standard \java{} compiler, and therefore must be processed with 
special tools. \eml{} (Extended ML) \cite{sannella-eml-97} takes a slightly different approach to specification and formal development of \sml{} programs.
\eml{} specifications look just like \sml{} programs except that axioms are allowed in signatures and in place of code in structures and functors. Some \eml{} specifications are executable, since \sml{} function definitions are just axioms of a certain special form. This makes EML a wide-spectrum language which can be used to express every stage in the development of a \sml{} program from the initial high-level specification to the final program itself and including intermediate stages in which specification and program are intermingled.

\notethis{References for Z and Clear.}.

\notethis{Other relevant literature, like C verification tools, diff tools, ...}

\notethis{What we do different.}


  
  
Model checking software:
  Java PathFinder 
    \cite{havelund-jpf-00}
    \cite{havelund-visser02}  
  \cite{holzmann-spin-2004}


\section{CONCLUSION}
\label{sec:conclusion}

We have presented an overview of the \Klang{} language in this
paper. \Klang{} is intended to be used in a modeling environment for
proving satisfiability of \sysml{} models and exploring solutions to
various types of specifications, such as structure,
planning/scheduling, etc. We have also presented in detail, our
methodology for performing automatic translation of \Klang{}
specifications to SMT-LIB, and using an SMT solver such as \zthree{}
to perform model finding. Using manual methods of creating \Klang{}
specifications from \sysml{} models and reference materials, we have
already observed \Klang{} provide value in the modeling environment by
discovering unsatisfiability of scheduling problems in the proposed
Europa Clipper mission concept, which was confirmed by external manual
analysis.  In our current experience, \Klang{} seems to be sufficient
for creating small to medium sized \sysml{} models and proving
properties about them.
%
Concerning problems faced, a main challenge of course is the
higher-order nature of K, requested by mission engineers
(expressiveness prioritized over guaranteed analyzability).  SMT-LIB
is generally first-order.  Some problems are a consequence of using
SMT-LIB solvers, which struggle with the combination of arrays (used
for the heap and for sets) and universal quantification. Additionally,
the use of Real numbers and arithmetic on them is also a known SMT
challenge, especially in the context of arrays.
%
We are now in the process of creating tools to automatically translate
\sysml{} models to \Klang{} specifications (and back) and perform analysis on
them using the \Klang{} infrastructure. This will make it possible to view
a model as graphics as well as in text. The translation of K needs to
be extended to cover more constructs, including statements with
side-effects. Other challenges is making K executable by translation
to Scala, including executing OCL-like expressions,
and providing support for reflection such that models can query themselves.



\bibliographystyle{abbrv}
\bibliography{biblio}

\appendix


\section{Grammar}
\label{sec:grammar}

\begin{verbatim}

grammar Model;

model:
    packageDeclaration?
    importDeclaration* 
    topDeclarationList? 
    EOF
  ;

topDeclarationList: topDeclaration (SEP topDeclaration)* SEP?
  ;

topDeclaration:
    memberDeclaration 
  | classDeclaration 
  | assocDeclaration 
  ;

packageDeclaration:
    'package' qualifiedName
  ;

importDeclaration:
    'import' qualifiedName ('.' '*')?
  ;

memberDeclarationList:
    memberDeclaration (SEP memberDeclaration)* SEP?
  ;

assocMemberDeclarationList:
    assocMemberDeclaration (SEP assocMemberDeclaration)* SEP?
  ;

classDeclaration:
    'class' Identifier typeParameters? extending? '{' memberDeclarationList? '}' 
  ;

assocDeclaration:
    'assoc' Identifier  '{' assocMemberDeclarationList? '}'
  ;

typeParameters:
      '[' typeParameter (',' typeParameter)* ']'
    ;

typeParameter:
      Identifier (':' typeBound)?
    ;

typeBound:
      type ('+' type)*
    ;
      
extending:
    'extends' type (',' type)*
  ;

type:
    primitiveType                   # PrimType
  | qualifiedName typeArguments?    # IdentType
  | type (tokenStar type)+          # CartesianType
  | type tokenArrow type            # FuncType
  | '{' type '}'                    # SetType
  | '[' type ']'                    # ListType
  | '<' type ',' type '>'           # MapType
  | '(' type ')'					# ParenType
  | '{|' Identifier ':' type  SUCHTHAT expression '|}' # SubType 
  | type '?' # OptionalType
  ;

expressionOrStar:
    expression
    | '*'
    ;

typeArguments:
    '[' type (',' type)* ']'
  ;

memberDeclaration:
    sortDeclaration 
  | typeDeclaration 
  | valueDeclaration 
  | variableDeclaration
  | functionDeclaration
  | constraint 
  | expression
  ;

assocMemberDeclaration:
    roleDeclaration
  | memberDeclaration
  ;

valueDeclaration:
    'val' pattern ('=' expression)? 
  ;

sortDeclaration:
    'type' Identifier 
  ;

typeDeclaration:
    'type' Identifier typeParameters? '=' type 
  ;

variableDeclaration:
    'var' pattern ('=' expression)? 
  ;

roleDeclaration:
    partDeclaration
  | refDeclaration
  ;

partDeclaration:
    'part' Identifier ':' Identifier multiplicity?
    ;

refDeclaration:
    'ref' Identifier ':' Identifier multiplicity?
    ;

multiplicity:
    expressionOrStar ('..' expressionOrStar)?
    ;

functionDeclaration:
    shortFunctionDeclaration
  | longFunctionDeclaration
  ;

shortFunctionDeclaration:
    'fun' Identifier ('(' patternList? ')')+ (':' type)? 
    '='
    expression
  ;

longFunctionDeclaration:
    'fun' Identifier ('(' patternList? ')')+ (':' type)? 
    functionSpecification*
    '{'
    memberDeclarationList?
    '}'
  ;

functionSpecification:
    'pre'  expression 
  | 'post' expression   
  ;

constraint:
    'req' (Identifier ':')?  expression
  ;

primitiveType:
    'Bool'
  | 'Char'
  | 'Int'       // Scala bigint (arbitrary precision)
  | 'Real'      // double
  | 'String'
  | 'Unit'  
  ;

tokenLessThan:
    '<' 
    | 'lt'
    ;
tokenGreatherThan:
    '>'
    | 'gt'
    ;
tokenLessThanEqual:
    '<='
| 'lte'
;
tokenGreaterThanEqual:
    '>='
    | 'gte'
    ;
tokenAnd:
    '&&' 
    | 'and'
    ;
tokenOr:
    '||'
    | 'or'
    ;
tokenNot:
    '!'
    | 'not'
    ;
tokenImplies:
    '=>'
    | 'implies'
    ;
tokenIFF:
    '<=>'
    | 'iff'
    ;
tokenEquals:
    '='
    | 'eq'
    ;
tokenStar:
    '*'
    ;
tokenArrow:
    '->'
    ;

tokenEnd: 
    'end' 
    ;
    
effect:
    expression (SEP expression)*
  ;
expression: 
    bracketedExpression # BracketedExp
  | literal #LiteralExp
  | Identifier #IdentExp
  | expression '.' Identifier #DotExp
  | expression '(' argumentList? ')' #AppExp
  | 'if' expression 'then' memberDeclarationList? ('else' memberDeclarationList?)? tokenEnd #IfExp
  | 'while' expression 'do' memberDeclarationList? tokenEnd #WhileExp
  | 'for' '(' pattern 'in' expression ')' 'do' memberDeclarationList? tokenEnd # ForExp 
  | 'match' expression 'with' match+ 'end' #MatchExp
  | tokenNot expression #NotExp
  | 'forall' rngBindingList SUCHTHAT expression #ForallExp 
  | 'exists' rngBindingList SUCHTHAT expression #ExistsExp 
  | '{' expressionList? '}' #SetEnumExp
  | '{' expression '..' expression '}' #SetRngExp
  | '{' expression '|' rngBindingList SUCHTHAT expression '}' #SetCompExp 
  | '[' expressionList? ']' #ListEnumExp
  | '[' expression '..' expression ']' #ListRngExp
  | '[' expression '|' pattern 'in' expression SUCHTHAT expression ']' #ListCompExp 
  | '<' mapPairList? '>' #MapEnumExp
  | '<' mapPair '|' rngBindingList SUCHTHAT expression '>' #MapCompExp 
  | expression ('*'|'/'|'%'|'inter'|'\\'|'++'|'#'|'^') expression #BinOp1Exp
  | expression ('+'|'-'|'union') expression #BinOp2Exp
  | expression 
      (
         tokenLessThanEqual | tokenGreaterThanEqual | tokenLessThan | tokenGreatherThan 
       | tokenEquals | tokenNot tokenEquals
       | 'isin'|'!isin'|'subset'|'psubset' 
      )  
    expression #BinOp3Exp
  | expression tokenAnd expression #AndExp
  | expression tokenOr expression #OrExp
  | expression (tokenImplies | tokenIFF) expression #IFFExp
  | expression ':=' expression #AssignExp
  | expression 'is' type # TypeCheckExp
  | expression 'as' type # TypeCastExp
  | 'assert' '(' expression ')' #AssertExp 
  | '~' expression #NegExp
  | pattern '->' expression #LambdaExp
  | 'continue' #ContinueExp
  | 'break' #BreakExp
  | 'return' expression? #ReturnExp
  | '$' #ResultExp
  ;

argumentList: 
    positionalArgumentList #PosArgList
  | namedArgumentList # NamedArgList
  ;

positionalArgumentList:
    expression (',' expression)* 
    ;

namedArgumentList:
   namedArgument (',' namedArgument)* 
  ;

namedArgument :
    Identifier '=' expression
  ;

bracketedExpression:
    '(' expression ')' #ParenExp
  | '(' expression (',' expression)+ ')' #TupleExp
  ;

idValueList:
    idValuePair (',' idValuePair)*
  ;

idValuePair:
    Identifier ':=' expression
  ;

match:
  'case' pattern ('|' pattern)* '=>' expression 
  ;

mapPairList:
    mapPair (',' mapPair)*
  ;

mapPair:
    expression ':' expression 
  ;

rngBindingList:
    rngBinding (',' rngBinding)*
  ;

rngBinding:
    patternList ':' collectionOrType
  ;

patternList:
    pattern (',' pattern)*
  ;

collectionOrType:
    expression
  | type
  ;
  
pattern:
    literal # LiteralPattern
  | '_' #DontCarePattern   
  | Identifier #IdentPattern
  | '(' pattern (',' pattern)+ ')' #CartesianPattern  
  | pattern ':' type # TypedPattern
  ;
  
identifierList:
    Identifier (',' Identifier)*
  ;

expressionList:
    expression (',' expression)*
  ;
    
qualifiedName:
    Identifier ('.' Identifier)*
  ;

literal:
    IntegerLiteral
  | FloatingPointLiteral
  | CharacterLiteral
  | StringLiteral
  | BooleanLiteral
  | NullLiteral
  | ThisLiteral
  ;

SUCHTHAT :
    '.' 
  ;

IntegerLiteral:
      DecimalIntegerLiteral
    |   HexIntegerLiteral
    |   OctalIntegerLiteral
    |   BinaryIntegerLiteral
    ;

fragment
DecimalIntegerLiteral:
      DecimalNumeral IntegerTypeSuffix?
    ;

fragment
HexIntegerLiteral:
      HexNumeral IntegerTypeSuffix?
    ;

fragment
OctalIntegerLiteral:
      OctalNumeral IntegerTypeSuffix?
    ;

fragment
BinaryIntegerLiteral:
      BinaryNumeral IntegerTypeSuffix?
    ;

fragment
IntegerTypeSuffix:
      [lL]
    ;

fragment
DecimalNumeral:
      '0'
    |   NonZeroDigit (Digits? | Underscores Digits)
    ;

fragment
Digits:
      Digit (DigitOrUnderscore* Digit)?
    ;

fragment
Digit:
      '0'
    |   NonZeroDigit
    ;

fragment
NonZeroDigit:
      [1-9]
    ;

fragment
DigitOrUnderscore:
      Digit
    |   '_'
    ;

fragment
Underscores:
      '_'+
    ;

fragment
HexNumeral:
      '0' [xX] HexDigits
    ;

fragment
HexDigits:
      HexDigit (HexDigitOrUnderscore* HexDigit)?
    ;

fragment
HexDigit:
      [0-9a-fA-F]
    ;

fragment
HexDigitOrUnderscore:
      HexDigit
    |   '_'
    ;

fragment
OctalNumeral:
      '0' Underscores? OctalDigits
    ;

fragment
OctalDigits:
      OctalDigit (OctalDigitOrUnderscore* OctalDigit)?
    ;

fragment
OctalDigit:
      [0-7]
    ;

fragment
OctalDigitOrUnderscore:
      OctalDigit
    |   '_'
    ;

fragment
BinaryNumeral:
      '0' [bB] BinaryDigits
    ;

fragment
BinaryDigits:
      BinaryDigit (BinaryDigitOrUnderscore* BinaryDigit)?
    ;

fragment
BinaryDigit:
      [01]
    ;

fragment
BinaryDigitOrUnderscore:
      BinaryDigit
    |   '_'
    ;

FloatingPointLiteral:
      DecimalFloatingPointLiteral
    |   HexadecimalFloatingPointLiteral
    ;

fragment
DecimalFloatingPointLiteral:
      Digits '.' Digits? ExponentPart? FloatTypeSuffix?
    |   '.' Digits ExponentPart? FloatTypeSuffix?
    |   Digits ExponentPart FloatTypeSuffix?
    |   Digits FloatTypeSuffix
    ;

fragment
ExponentPart:
      ExponentIndicator SignedInteger
    ;

fragment
ExponentIndicator:
      [eE]
    ;

fragment
SignedInteger:
      Sign? Digits
    ;

fragment
Sign:
      [+-]
    ;

fragment
FloatTypeSuffix:
      [fFdD]
    ;

fragment
HexadecimalFloatingPointLiteral:
      HexSignificand BinaryExponent FloatTypeSuffix?
    ;

fragment
HexSignificand:
      HexNumeral '.'?
    |   '0' [xX] HexDigits? '.' HexDigits
    ;

fragment
BinaryExponent:
      BinaryExponentIndicator SignedInteger
    ;

fragment
BinaryExponentIndicator:
      [pP]
    ;

BooleanLiteral:
      'true'
    |   'false'
    ;

NullLiteral:
  'null'
  ;

ThisLiteral:
  'this'
  ;

CharacterLiteral:
      '\'' SingleCharacter '\''
    |   '\'' EscapeSequence '\''
    ;

fragment
SingleCharacter:
      ~['\\]
    ;
    
StringLiteral:
      '"' StringCharacters? '"'
    ;

fragment
StringCharacters:
      StringCharacter+
    ;

fragment
StringCharacter:
      ~["\\]
    |   EscapeSequence
    ;

fragment
EscapeSequence:
      '\\' [btnfr"'\\]
    |   OctalEscape
    |   UnicodeEscape
    ;

fragment
OctalEscape:
      '\\' OctalDigit
    |   '\\' OctalDigit OctalDigit
    |   '\\' ZeroToThree OctalDigit OctalDigit
    ;

fragment
UnicodeEscape:
      '\\' 'u' HexDigit HexDigit HexDigit HexDigit
    ;

fragment
ZeroToThree:
      [0-3]
    ;

Identifier:
      JavaLetter JavaLetterOrDigit*
    ;

fragment
JavaLetter:
      [a-zA-Z$_] // these are the "java letters" below 0xFF
    |   // covers all characters above 0xFF which are not a surrogate
        ~[\u0000-\u00FF\uD800-\uDBFF]
        {Character.isJavaIdentifierStart(_input.LA(-1))}?
    |   // covers UTF-16 surrogate pairs encodings for U+10000 to U+10FFFF
        [\uD800-\uDBFF] [\uDC00-\uDFFF]
        {Character.isJavaIdentifierStart(Character.toCodePoint((char)_input.LA(-2), (char)_input.LA(-1)))}?
    ;

fragment
JavaLetterOrDigit:
      [a-zA-Z0-9$_] // these are the "java letters or digits" below 0xFF
    |   // covers all characters above 0xFF which are not a surrogate
        ~[\u0000-\u00FF\uD800-\uDBFF]
        {Character.isJavaIdentifierPart(_input.LA(-1))}?
    |   // covers UTF-16 surrogate pairs encodings for U+10000 to U+10FFFF
        [\uD800-\uDBFF] [\uDC00-\uDFFF]
        {Character.isJavaIdentifierPart(Character.toCodePoint((char)_input.LA(-2), (char)_input.LA(-1)))}?
    ;

fragment // ;; added this
CommentBegin:
    '---' '-'*
   | '===' '='* // to experiment with different ways of showing start of comment
   ;

fragment 
CommentEnd:
    '---' '-'*
   ;

COMMENT :
     CommentBegin .*? CommentEnd -> skip
  ;

LINE_COMMENT:
    '--' ~[\r\n]* -> skip
  ;

WS:
    [ \t\r\n\u000C]+ -> skip
  ;

SEP:
    ';'
  ;

SEPSEP:
    ';;'
  ;

\end{verbatim}


\end{document}
