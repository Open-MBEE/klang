\section{CONCLUSION}
\label{sec:conclusion}

We have presented an overview of the \Klang{} language in this
paper. \Klang{} is intended to be used in a modeling environment for
proving satisfiability of \sysml{} models and exploring solutions to
various types of models, such as structure,
planning/scheduling, etc. We have also presented in detail, our
methodology for performing automatic translation of \Klang{}
models to SMT-LIB, and using an SMT solver such as \zthree{}
to perform semantic model finding. Using manual methods of creating \Klang{}
models from \sysml{} models and reference materials, we have
already observed \Klang{} provide value in the modeling environment by
discovering unsatisfiability of scheduling problems in the proposed
Europa Clipper mission concept, which was confirmed by external manual
analysis.  In our current experience, \Klang{} seems to be sufficient
for creating small to medium sized \sysml{} models and proving
properties about them.
%
Concerning problems faced, a main challenge of course is the
higher-order nature of K, requested by mission engineers
(expressiveness prioritized over guaranteed analyzability).  SMT-LIB
is generally first-order.  Some problems are a consequence of using
SMT-LIB solvers, which struggle with the combination of arrays (used
for the heap and for sets) and universal quantification. Additionally,
the use of Real numbers and arithmetic on them is also a known SMT
challenge, especially in the context of arrays.
%
We are now in the process of creating tools to automatically translate
\sysml{} models to \Klang{} models (and back) and perform analysis on
them using the \Klang{} infrastructure. This will make it possible to view
a model as graphics as well as in text. The translation of K needs to
be extended to cover more constructs, including statements with
side-effects. Other challenges include making K executable, for example by translation
to Scala, including executing OCL-like expressions;
providing support for reflection such that models can query themselves;
and making the K language and textual notation user-extensible.
