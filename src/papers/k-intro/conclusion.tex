
\section{Conclusion}
\label{sec:conclusion}

We have presented an overview of the \Klang{} language in this
paper. \Klang{} is intended to be used in a modeling environment for
proving satisfiability of \sysml{} models and exploring solutions to
various types of specifications, such as structure,
planning/scheduling, etc. We have also presented in detail, our methodology
for performing automatic translation of \Klang{} specifications to
SMT-LIB, and using an SMT solver such as \zthree{} to perform model
finding. Using manual methods of creating \Klang{} specifications from
\sysml{} models and reference materials, we have already observed
\Klang{} provide value in the modeling environment by discovering
unsatisfiability of scheduling problems in the proposed Europa Clipper
mission concept, which was confirmed by external manual analysis.  Due
to JPL's information release restrictions, such details cannot be
shared.  In our current experience, \Klang{} seems to be sufficient
for creating small to medium sized \sysml{} models and proving
properties about them.
%
Concerning problems faced, a main challenge of course is the
higher-order nature of K, requested by mission engineers
(expressiveness prioritized over guaranteed analyzability). 
SMT-LIB is generally
first-order.  Some problems are a consequence of using SMT-LIB
solvers, which struggle with the combination of arrays (used for the
heap and for sets) and unversal quantification. Real arithmetic is a
known SMT challenge, especially in the context of arrays.
%
We are now in the process of creating tools to automatically translate
\sysml{} models to \Klang{} specifications (and back) and perform analysis on
them using the \Klang{} infrastructure. This will make it possible to view
a model as graphics as well as in text. The translation of K needs to
be extended to cover more constructs, including statements with
side-effects. Other challenges is making K executable by translation
to Scala, including executing OCL-like expressions,
and providing support for reflection such that models can query themselves.

