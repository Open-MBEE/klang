
\section{Conclusion}
\label{sec:conclusion}

We have presented an overview of the \Klang{} language in this
paper. \Klang{} is intended to be used in a modeling environment for
proving satisfiability of \sysml{} models and exploring solutions to
various types of specifications, such as structure,
planning/scheduling, etc. We also present in detail, our methodology
for performing automatic translation of \Klang{} specifications to
SMT-LIB and using an SMT solver such as \zthree{} to perform model
finding. Using manual methods of creating \Klang{} specifications from
\sysml{} models and reference materials, we have already observed
\Klang{} provide value in the modeling environment by discovering
unsatisfiability of scheduling problems in the Europa project, which
was confirmed by external manual analysis. Due to information
classification restrictions, these details cannot be shared at this
current time. In our current experience, \Klang{} seems to be
sufficient for creating small to medium sized \sysml{} models and
proving properties about them. We are now in the process of creating
tools to automatically translate the \sysml{} models for the NASA
Europa project to \Klang{} specifications and perform analysis on them
using the \Klang{} infrastructure. We believe that successfully
dealing with inheritance will be the key for \Klang{} to be provide
value when analyzing \sysml{} models.

