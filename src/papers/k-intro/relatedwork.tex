
\section{Related Work}
\label{sec:related-work}

\Klang{} is intended to represent a textual modeling language capable
of representing \sysml{} concepts, specifically class diagrams with
constraints.  However, as mentioned in the introduction, it also
contains programming constructs.  As such it can be perceived as a
wide-spectrum modeling/programming language.

Wide spectrum specification languages have been investigated to length
in the formal methods community. One of the well-known examples is
\vdm{} \cite{vdm78,bjoerner-jones-82,jones90,jones-shaw-90}. \vdm{} in
its original form \cite{vdm78} provided a combination of procedural
programming and functional programming, as well as specification using
sets, lists and maps (with proper mathematical notation), and
higher-order predicate logic. \vdmpp{} \cite{vdmplusplus05} added
object-orientation to \vdm{}, which is now part of the \vdm{}
standard. The \raiselang{} specification language (\rsl{})
\cite{raise92} is a wide-spectrum language taking inspiration from
\vdm{} as well as from other modeling languages such as \zlang{}
\cite{spivey-Z-1988}, and from algebraic equational specification
languages. Here refinement is the simpler theory implication: the
implementation shall imply the specification in a logic sense. \asml{}
\cite{asml05} is a more recent wide-spectrum specification language,
in many ways similar to \vdm{}, but based on the fundamental concept
that operations operate on algebras.  Other fundamental works on
refinement include (not a comprehensive list):
\cite{wirth-refinement-71,hoare-sanders-refinement-86,morgan-refinement-94,woodcock-sanders-z-96,back-wright-refinement-98,abrial-eventb-10}.

\alloy{} \cite{jackson-alloy-12} added new life to this community by
being supported by a automated SAT solver. In many respects, \Klang{}
is close in spirit to \alloy{}, but differs by being supported by an
automated SMT solver (in contrast to a SAT solver), resulting in a
richer set of constructs, including arithmetic, being exposed to
analysis. \Klang{} also combines a type view as found in traditional
specification and programming languages, as well as a relational view,
whereas \alloy{} is purely relational. We are of the belief that the
notion of a type is fundamental to programming as well as to
modeling. In contrast to automated provers, interactive theorem
provers such as \pvs{} \cite{cade92-pvs,pvs-website}, \coq{}
\cite{coq-website}, and \isabelle{} \cite{isabelle-website}, allow the
user to steer the proofs.  Although this allows to perform more
complex proofs, it is also requires more skills of the user, and time,
which is often a limited resource in software development projects.

Several high-level programming languages have been developed over
time, including the early \sml{} (Standard ML) \cite{standard-ml-97},
its derivative \ocaml{} \cite{ocaml}, and \haskell{}
\cite{haskell}. However, also \java{} can be considered high-level due
to its libraries of collections (sets, lists, and maps), as well as
the iterator concept. \python{} \cite{python} is close to combining
object-oriented and functional programming. \scala{} \cite{scala} does
this to the full extent. The close relationship between \scala{} and
\vdm{} is discussed in \cite{havelund-scala-vdm-12}.  \fortress{}
\cite{fortress} introduced built-in notation for sets, lists, and
maps, very much resembling the notation in \vdm{}.

Specification constructs have been introduced in programming
languages, in the form of design-by-contract (pre/post conditions +
class invariants). Examples are \eiffel{} \cite{eiffel} and
\specsharp{} \cite{specsharp}, where contracts are part of the
language. \scala{} has library functions for writing pre/post
conditions on functional programs \cite{odersky-rv10}. Finally, The
\jml{} language \cite{jml} allows to write design-by-contract
specifications for \java{} as comments. These are ignored by the
standard \java{} compiler, and therefore must be processed with
special tools. \eml{} (Extended ML) \cite{sannella-eml-97} takes a
slightly different approach to specification and formal development of
\sml{} programs.  \eml{} specifications look just like \sml{} programs
except that axioms are allowed in signatures and in place of code in
structures and functors. Some \eml{} specifications are executable,
since \sml{} function definitions are just axioms of a certain special
form. This makes \eml{} a wide-spectrum language.

Programming languages are now also being built with verification in
mind.  \dafny{} \cite{leino-lpar-2010} supports specifications that
can be used to write functional-correctness conditions for programs.
It is supported by verifier, which is implemented on top of the
\boogie{} verification engine, which itself is built on top of
\zthree.  \whythree{} \cite{filliatre-why3-2011} provides a rich
language for specification and programming, called \whyml{}, and
relies on external theorem provers, both automated and interactive, to
discharge verification conditions. A user can write \whyml{} programs
directly and get correct-by-construction \ocaml{} programs through an
automated extraction mechanism. Model checking is another form of
analysis that has been applied to programming languages.  Java
PathFinder~\cite{havelund-jpf-00,havelund-visser02} performs model
checking of Java programs. \slam{} \cite{ball2010slam2} performs
static analysis and counter-example guided abstraction refinement to
device drivers, and has been applied in a large scale industry
setting.  \spin{} \cite{holzmann-spin-2004} performs model checking of
models expressed in the \promela{} language, but can also model check
\clang{} code directly.

The great improvements in model checking, static analysis, theorem
proving, and SMT solvers such as Z3~\cite{de2008z3} have all
contributed to investigating and dealing with software change. To this
effect, differential symbolic execution~\cite{person2008differential}
has been investigated for establishing equivalence between two
versions of a program. The work described in \cite{lahiri2012symdiff}
uses verification conditions and SMT solvers for detecting semantic
change between two closely related versions of a function (program),
by discovering inputs to the function that cause the outputs to
differ. The work described in \cite{godlin2009regression} deals with
regression verification and provides a technique for performing
equivalence checking of C programs, by using the older version of the
program as a specification for the new version of the program. A large
part of the inspiration for such work comes from the theorem proving
community.

An important use of \Klang{} that we have observed so far, which
differs in the way traditional verification tools are used, is that
modelers tend to use \Klang{} along with it's solving ability as a
tool for {\em discovering} the right set of constraints for their
class before introducing a change. For example, uncertainty about a
particular variable and it's potential range of valid values can be
quite common in modeling environments. Since \Klang{} helps discover
unsatisfiability, modelers use an iterative refinement technique to
discover the appropriate range of a variable for their needs. \Klang{}
in this case is providing validation before a change is completely
committed.


