\section{RELATED WORK}
\label{sec:related-work}

\Klang{} is a wide spectrum textual language containing concepts such
as classes, inheritance, properties, functions, and
expressions. Expressions in \Klang{} are very rich and provide a basis
for expressing any higher order logic formula succinctly. The language
also contains constructs such as lists, sets, loops etc. for
performing general sequential programming, functional programming, and
object oriented programming. The main goal is to automatically
translate \sysml{} diagrams and models to \Klang{} and perform various
types of analysis such as type checking, satisfiability checking, and
potentially execution.

The formal methods community has studied and investigated wide
spectrum specification languages in
detail.~\cite{vdm78,bjoerner-jones-82,jones90,jones-shaw-90} present
the \vdm{} language, which provides a combination of procedural and
functional programming along with sets, lists, maps, and higher order
predicate logic in proper mathematical notation. A natural evolution
of \vdm{} was to introduce object orientation, as shown in
\vdmpp{}~\cite{vdmplusplus05}. The \raiselang{} specification
language~\cite{raise92} is another wide spectrum language that takes
inspiration from \vdm{}, \zlang{}~\cite{spivey-Z-1988} and algebraic
specification languages. More recently, \asml{} presented
in~\cite{asml05} presents a language where all operations operate on
algebras. \alloy{}~\cite{jackson-alloy-12} is a language intended for
describing structure and exploring the design of the structure through
the specification of constraints. \alloy{} also provides seamless
integration with a SAT solver. FORMULA~\cite{jackson2009specifying}
also presents a language for logic programming where the model is then
analyzed by SMT solvers. FORMULA focuses on requirements for
incremental refinement purposes and design exploration.

In contrast to specification languages, high level programming
languages have also been developed over time. \sml{} (Standard ML)
\cite{standard-ml-97}, \ocaml{} \cite{ocaml}, and \haskell{}
\cite{jones2003haskell} belong to the same family of functional
programming languages with some support for classes and object
orientation. \python{} \cite{python} also provides a unique
perspective on combining object oriented and functional
programming. In our view, \scala{} \cite{scala} does so to the fullest
extent. The close relationship between \scala{} and \vdm{} is
discussed in \cite{havelund-scala-vdm-12}.  \fortress{}
\cite{fortress} introduced built-in notation for sets, lists, and
maps, very much resembling the notation in \vdm{}.

Yet another class of recent languages has investigated introducing
specification constructs in programming languages in the form of
design-by-contract (pre/postconditions and class invariants). Examples
of such languages include \eiffel{} \cite{meyer1988eiffel} and
\specsharp{} \cite{barnett2011specification}.  \scala{} also attempts
at providing such constructs, but through the use of library functions
on functional programs \cite{odersky-rv10}. Finally, The \jml{}
language \cite{leavens1998jml} allows to write design-by-contract
specifications for \java{} as comments. These specifications and
constructs are ignored by the standard \java{} compiler. Rather, they
are processed with special tools; thus, in some cases making them less
valuable to the language.

With the great improvements being made to SAT and SMT
solvers~\cite{smt-lib,de2008z3}, programming languages now are also
built with verification as a primary goal. \dafny{}
\cite{leino-lpar-2010} provides explicit support for specifications in
the program that can be used to specify functional correctness
constraints for programs. These constraints and specifications are
verified using the \boogie{}~\cite{barnett2006boogie} verifier that
uses \zthree{} as the underlying SMT solver. \dafny{} is an excellent
language for performing verification, but lacks support for
inheritance and class invariants, which are a necessity for the kind
of models we deal with. Similarly, \specsharp{} also provides support
for verifying the user provided specifications. \whythree{}
\cite{filliatre-why3-2011} provides a rich language for specification
and programming, called \whyml{}. \whythree{} relies on external
theorem provers (automatic and interactive) to then verify the
specification. Additionally, using \whyml{}, one can also generate
\ocaml{} programs using a correct by construction automatic extraction
procedure. In contrast to using automated provers, interactive theorem
provers such as \pvs{} \cite{cade92-pvs,pvs-website}, \coq{}
\cite{barras1997coq}, and \isabelle{} \cite{nipkow2002isabelle}, also
provide languages to create specifications and provide constraints,
which can then be discharged via a user guided process. Such tools
allow for proving much more complex properties, but tend to be
laborious and non-trivial to use.

Past research in formal modeling has also resulted in various attempts
at combining formal and semi-formal languages. Work
in~\cite{lausdahl2009connecting} presents an approach to translate
between UML and \vdmpp{}, and~\cite{kim2005mda} presents work on
integrating a formal language such as Object-Z with UML in a single
combined framework. To the best of our knowledge, these approaches
have not focused on translating constraints and integrating with an
SMT solver, and have been primarily focused on UML (not \sysml).

\Klang{} is very close in spirit and ideology to many of the
aforementioned languages, but differs in many respects as well. For
example, due to the application environment of \Klang{} being targeted
to \sysml{} models, proving class satisfiability and model finding are
of prime importance, something for which \Klang{} is optimized. \Klang{}
also provides support for specifying and proving specifications with
multiple inheritance, something \sysml{} models depend on
greatly. Integration with SMT solvers has proven to be useful not only
for proving class satisfiability, but also for model finding
(including scheduling) and {\em model exploration}. In model
exploration, users manually explore the range of satisfiable solutions
for the given model using iterative refinement techniques (changing
constraints manually).
