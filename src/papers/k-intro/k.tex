\section{K Syntax}
\label{sec:k-syntax}

In this section we introduce the syntax of \Klang{}. We use the
\Klang{} example in Figure~\ref{fig:fs} as our running example for
discussing core concepts in \Klang{}. The example shows a basic model
of a flash file system modeled using \Klang{}. It is intended to be a
basis for discussing language features, and not a complete model of
the flash file system. Appendix~\ref{app:grammar} shows the ANTLR
grammar for \Klang{} (with trivial parts removed).

\begin{figure}
\centering
\begin{tabular}{c}
\hline \\
\lstinputlisting{examples/fs.k} \\ \\
\hline
\end{tabular}
\caption{A simple model of a spacecraft using \Klang{}.}
\label{fig:fs}
\end{figure}

\Klang{} is high level textual language which supports multiple
paradigms. It allows one to create \name{packages}, \name{classes},
and \name{functions}. Packages are collections of classes, which can
then be \name{imported} by other \Klang{} files. Line 1 in
Figure~\ref{fig:fs} shows an example of a package
declaration. Classes, similar to other languages provide a means for
abstracting and collecting properties. In \Klang{} classes may contain
functions, properties, and constraints (requirements). Lines 8 -- 11
in Figure~\ref{fig:fs} declare an \code{Entry} class, which contains
two members: property \code{name} of type \code{String}, and function
\code{size} that takes no arguments and returns an \code{Int}. For
function \code{size}, the function implementation is not specified for
function \code{size}. \code{String} is one of the six primitive types
provided by \Klang{}: \code{Int}, \code{Real}, \code{String},
\code{Char}, \code{Unit}, and \code{Boolean}. \Klang{} also provides
the following collections:

\begin{description}
\item [Bag:] collection of items not subject to any order
  or uniqueness constraints.
\item [Seq:] collection of items subject to an ordering, but
  no uniqueness constraints.
\item [Set:] collection of items subject to uniqueness
  constraints, but no ordering.
\end{description}

\Klang{} provides {\em subtypes}. Line 5 specifies a subtype
named \code{Byte}, which is derived from the \code{Int} type but
restricted to values between 0 and 256.

\Klang{} allows classes to inherit from one or more classes. For
example, class \code{Dir}, specified on lines 13 -- 18 extends the
\code{Entry} class. As with other languages, inheritance causes the
child classes to inherit the instance variables and functions of the
parent classes, but in addition, in \Klang{}, the child classes also
inherit the constraints from the parent classes. In the case of
multiple inheritance, \Klang{} requires that the property names be
unique across all classes. Functions on the other hand may be
overloaded by changing the function signature. Both class \code{File}
and \code{Dir} inherit from class \code{Entry} and define a body for
the \code{size} function. Lines 15 -- 17 in Figure~\ref{fig:fs}
provide the \code{size} function for the \code{Dir} class. It makes
use of the \code{sum} function, that is provided by \Klang{} for all
collections. This is the same function as declared in class
\code{Entry}. Currently, function bodies cannot be declared more than
once along an inheritance path. Functions may take an arbitrary number
of arguments and return a single value. \Klang{} also provides
\code{Tuple} to group objects together. On line 29, we see a
constraint being specified for class \code{Block} using the \code{req}
keyword. The constraint specifies that the size function of
\code{Block} should always return the same value as specified in the
global property \code{SIZE\_OF\_BLOCK} (left unspecified). Any number
of constraints can be specified at the global scope or within classes. 

In class \code{FileSystem}, two functions \code{mkDir} and
\code{rmDir}. The \code{mkDir} function takes a single argument
(\code{n} of type \code{String}) and returns a \code{FileSystem}
object which contains one additional directory entry that has name
\code{n}. The \code{rmDir} function has no body specified. Both
functions are defined along with a {\em function
  specification}. Function specifications are a list of {\em pre} and
{\em post} conditions that describe the precondition and postcondition
of the function. Any number of specifications may be provided. Line 38
specifies the precondition for function \code{mkDir} with the use of
an {\em existential} quantifier. It specifies that when creating a
directory in the file system, the given name \code{n} should not exist
in the current set of entries in the file system. \Klang{} provides
both {\em existential} and {\em universal} quantification in it's
expression language. For the same function, line 39 specifies the
postcondition. \code{\$result} is a reserved word that refers to the
return value of the function. It can only be used when specifying
postconditions. The post condition for \code{mkDir} specifies that
after the function \code{mkDir} has completed execution, there should
be no change in the resulting \code{FileSystem} object relative to the
current \code{FileSystem} object, which was used to create the new
directory. Line 41, which is the body of function \code{mkDir} creates
a new \code{FileSystem} object by calling the constructor for class
\code{FileSystem}. The only argument provided to the constructor is a
\code{Dir} object which contains one additional \code{Dir} entry whose
name is \code{n}. \Klang{} provides constructors automatically for all
classes where the arguments are {\em named arguments}. Each named
argument is of the form \code{member :: value} where the \code{::}
notation is used as a form of assignment. Multiple named arguments can
be provided as a comma delimited list. It is not necessary to specify
a value for all members of a class. Any members that are specified in
a constructor call will be assigned the specified value, and the rest
will be given the {\em default} value for their type. Function
\code{rmDir} is specified with no body, but only function
specifications. The function specifications require that function
\code{rmDir} only execute if the provided directory \code{n} exists in
the current object's contents. The postcondition specifies that the
resulting \code{FileSystem} object should be either the same size or
smaller relative to the current object.

Expressions in \Klang{} are the core of the language. Logical
expressions in \Klang{} allow one to write assignments, binary
expressions (such as and, or, implication, and iff), logical negation,
arithmetic negation, quantification, \code{is} for checking type, and
\code{as} for type casting. Any expression can use variables, function
application, lambda functions, and dot expressions. \Klang{} also
supports control expressions such as \code{if-then-else},
\code{while}, \code{match}, \code{for}, \code{continue}, \code{break},
and \code{return}. These expressions are similar to control
expressions provided in other languages such as Scala or Java. A
detailed description of the expression language is beyond the scope of
this paper. Appendix~\ref{app:grammar} provides the complete grammar
for the expression language.

\Klang{} also provides {\em multiplicities} as part of the
language. The concept is highly influenced by multiplicities that can
be found in other languages such as UML/SysML. The goal with
multiplicities is to be able to specify the size of a collection. For
example, Figure~\ref{fig:mult} shows a \Klang{} model of a
\code{Person} that can has various member properties and the
corresponding inferred types for each member property. We will analyze
each of these individually.

\sysml{} models also carry a lot of metadata information in them. To
accommodate for this, \Klang{} also provides the {\em annotation}
construct. New annotations can be created and applied to classes,
expressions, functions etc. Currently, each annotation has a name and
a type. 

\begin{figure}
\caption{Example model and inferred types for members of class \code{Person}.}
\centering
\begin{tabular}[c]{c|c}
\begin{subfigure}[c]{0.5\textwidth}
\hspace{1cm}\scalebox{0.8}{\lstinputlisting{examples/mult.k} }
\end{subfigure}
&
\begin{subfigure}[c]{0.5\textwidth}
%\includegraphics[scale=0.35]{mult.png}
\hspace{1cm}\scalebox{0.8}{\lstinputlisting{examples/multr.k} }
\end{subfigure}
\\
\end{tabular}
\label{fig:mult}
\end{figure}

Each \code{Person} can have exactly one \code{mother}. This is
specified by line 4. No explicit multiplicity is specified, but in
\Klang{}, this is assumed to be a multiplicity of 1. A \code{Person}
can also have children, which is specified by line 5. Line 6 specifies
that a \code{Person} may have \code{cars}, which semantically
translates to a \code{Bag} (\Klang{} default) of \code{Car}, but is
written using a multiplicity (as opposed to the use of a
collection). Finally, a person may own many \code{accnts}, which is a
\code{Set} of \code{BankAccount}, specified to have a multiplicity of
1 or more. This translates to \code{accnts} being a \code{Bag} of
\code{Set[BankAccount]} with at least 1 entry and no upper limit. 

\subsection{K Type Checking}

The \Klang{} type checker performs basic checks on the provided input
to ensure naming and type consistency. It is used to ensure that all
declarations, expressions, annotations etc. are logically sound and
reference names (functions, members, variables) that exist and are
type consistent in the given context. Type information for all
expressions and any other inferences made by the type checker are
saved and made available to all other analyses/modules in the \Klang{}
tool chain. Further, when \Klang{} is asked to do analysis of the
input using SMT, the type checker imposes a stricter set of rules on
the provided input to ensure that it can be completely and correctly
translated to SMT. More details are provided in
Section~\ref{sec:k2smt}. The type checker is implemented as a stand
alone module, which is invoked after the AST has been constructed by a
visitor (interfacing with ANTLR). The implementation is done using
Scala.

