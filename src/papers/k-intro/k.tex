\section{K Syntax}
\label{sec:k-syntax}

In this section we introduce the syntax of \Klang{}. We use the
\Klang{} example in Figure~\ref{fig:fs} as our running example for
discussing core concepts in \Klang{}.

\begin{figure}
\centering
\begin{tabular}{c}
\hline \\
\lstinputlisting{examples/fs.k} \\ \\
\hline
\end{tabular}
\caption{A simple model of a spacecraft using \Klang{}.}
\label{fig:fs}
\end{figure}

\Klang{} is high level textual language which supports multiple
paradigms. It allows one to create {\em packages}, {\em classes}, and
{\em functions}. Packages are collections of classes, which can then
be {\em imported} by other \Klang{} files. Line 1 in
Figure~\ref{fig:fs} shows an example of a package
declaration. \Klang{} inherently provides the following collections:

\begin{description}
\item [Bag] A collection of items that is not subject to any order
  or uniqueness constraints.
\item [Seq] A collection of items that is subject to an ordering, but
  not any uniqueness constraints.
\item [Set] A collection of items that is subject to uniqueness
  constraints, but not any specific ordering.
\end{description}





\begin{enumerate}
\item Classes
\item Inheritance
\item Functions
\item Constraints
\item undefined functions
\item Sequences and Sets
\item Constructor with named arguments
\item Primitive types
\item Expressions
\item Function specifications
\item Quantification
\item Bounded types
\end{enumerate}

\subsection{K Type Checking}

Will use the same example as presented for the syntax.
