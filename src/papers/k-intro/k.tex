\section{INTRODUCTION TO \Klang{}}
\label{sec:k-syntax}

In this section we introduce the \Klang{} language. We use the
\Klang{} model in Figure~\ref{fig:fs} as our running example for
discussing core concepts in \Klang{}. The example shows a model of a
file system modeled using \Klang{}. It is intended to be a basis for
discussing language features, and not a complete model of a flash file
system.

\begin{figure}
  \centering
  \begin{tabular}{c}
    %\hline \\
    \small
    \lstinputlisting{examples/fs.k}
    % \hline
    \end{tabular}
  \vspace{0.2cm}
  \caption{A simple model of a file system using \Klang{}.}
  \label{fig:fs}
\end{figure}
  
\Klang{} is a high level textual language which supports multiple
paradigms. It allows one to create \name{packages}, which are
collections of \name{classes}. Packages can be \name{imported} by
other \Klang{} files. Line 1 in Figure~\ref{fig:fs} shows an example
of a package declaration. Classes, as in other object-oriented
languages, provide a means for abstracting and grouping properties
(member instance variables). In \Klang{}, classes may contain
properties, functions (there is no distinction between functions and
methods), and constraints (requirements). Scoping rules in \Klang{}
are similar to languages such as Java and C++. Lines 9 -- 12 in
Figure~\ref{fig:fs} declare an \code{Entry} class, which contains two
members: property \code{name} of type \code{String}, and function
\code{size} that takes no arguments and returns an \code{Int}. The
function implementation is not specified for function
\code{size}. \code{String} is one of the six primitive types provided
by \Klang{}: \code{Int}, \code{Real}, \code{String}, \code{Char},
\code{Unit}, and \code{Boolean}. \Klang{} also provides the following
collections:

\begin{description}
\item [Bag:] collection of items not subject to any order
  or uniqueness constraints.
\item [Seq:] collection of items subject to an ordering, but
  no uniqueness constraints.
\item [Set:] collection of items subject to uniqueness, but no
  ordering constraints.
\end{description}

\noindent \Klang{} provides {\em predicate subtypes}. Line 5 specifies
a subtype named \code{Byte}, which is derived from the \code{Int} type
but restricted to values between 0 and 256. \Klang{} allows classes to
inherit from one or more classes. For example, class \code{Dir},
specified on lines 14 -- 20 extends the \code{Entry} class. As with
other languages, inheritance causes the child classes to inherit the
instance variables and functions of the parent classes, but in
addition, in \Klang{}, the child classes also inherit the constraints
from the parent classes. In the case of multiple inheritance, \Klang{}
requires that the property names be unique across all
classes. Functions on the other hand may be overloaded by changing the
function signature. Both class \code{File} and \code{Dir} inherit from
class \code{Entry}. Lines 16 -- 19 in Figure~\ref{fig:fs} show the
implementation for the \code{size} function in the \code{Dir}
class. It makes use of the \code{sum} function, that is provided by
\Klang{} for all numerical collections. The \code{size} function is
the same as declared in class \code{Entry}. Currently, function bodies
cannot be declared more than once along an inheritance path. Functions
may take an arbitrary number of arguments and return a single
value. \Klang{} also provides tuples to group objects together. On
line 32, we see a constraint being specified for class \code{Block}
using the \code{req} (require) keyword. The constraint specifies that
the size function of \code{Block} should always return a value that is
less than or equal to the value specified in the global property
\code{SIZE\_OF\_BLOCK} (left unspecified). Any number of constraints
can be specified at the global scope or within classes. Constraints
are Boolean expressions, that restrict the values variables can
take. Constraints in a class can be considered class invariants.

Class \code{FS} (for \code{FileSystem}) contains two functions:
\code{mkDir} and \code{rmDir}. The \code{mkDir} function takes a
single argument (\code{n} of type \code{String}) and returns a
\code{FS} object which contains one additional directory entry that
has name \code{n}. The \code{rmDir} function has no body
specified. Both functions are defined along with a {\em function
  specification}. Function specifications are a list of {\em pre} and
{\em post} conditions that describe the precondition and postcondition
of the function. Any number of specifications may be provided. Line 39
specifies the precondition for function \code{mkDir} with the use of
an {\em existential} quantifier. It specifies that when creating a
directory in the file system, the given name \code{n} should not exist
in the current set of entries in the file system. \Klang{} provides
both {\em existential} and {\em universal} quantification in its
expression language. For the same function, line 42 specifies the
postcondition. \code{\$result} is a reserved word that refers to the
return value of the function. It can only be used when specifying
postconditions. The post condition for \code{mkDir} specifies that
function \code{mkDir} returns a \code{FS} object that has the same
size as the current \code{FS} object, which was used to create the new
directory. Line 44, which is the body of function \code{mkDir} creates
a new \code{FS} object by calling the constructor for class
\code{FS}. The only argument provided to the constructor is a
\code{Dir} object which contains one additional \code{Dir} entry whose
name is \code{n}. \Klang{} provides constructors automatically for all
classes where the arguments are {\em named arguments}. Each named
argument is of the form \code{member :: value} where the `\code{::}'
notation is used as a form of assignment. Multiple named arguments can
be provided as a comma delimited list. It is not necessary to specify
a value for all members of a class. Any members that are specified in
a constructor call are assigned the specified value, and the rest are
left underspecified.  Line 48 invokes the \code{FS} constructor as
well as the \code{Dir} constructor to create a new \code{FS} object
that contains an additional directory. Function \code{rmDir} is
specified with no body, but only function specifications. The function
specifications require that function \code{rmDir} only execute if the
provided directory \code{n} exists in the current object's
contents. The postcondition specifies that the resulting \code{FS}
object should be either the same size or smaller relative to the
current object.

Expressions in \Klang{} are the core of the language. Logical
expressions in \Klang{} allow one to write assignment statements
(side-effects), binary expressions (such as and, or, implication, iff
etc.), logical negation, arithmetic negation, quantification,
\code{is} for checking type, and \code{as} for type casting. Any
expression can use variables, function application, lambda functions,
and dot expressions. \Klang{} also supports control expressions such
as \code{if-then-else}, \code{while}, \code{match}, \code{for},
\code{continue}, \code{break}, and \code{return}. These expressions
are similar to control expressions provided in other languages such as
Scala or Java. A detailed description of the expression language is
beyond the scope of this paper. 

\Klang{} also provides {\em multiplicities} as part of the
language. Multiplicities in \Klang{} are greatly influenced by similar
concepts in graphical formalisms such as UML/SysML. In \Klang{},
multiplicities can be used as a short hand for specifying collections
and also restricting the size of collections. Figure~\ref{fig:mult}
shows a \Klang{} model of a \code{Person} that has various member
properties, and the corresponding inferred type for each member
property. We will analyze each of these individually.

\begin{figure*}
  \centering
  \begin{tabular}[c]{c|c}
    \begin{subfigure}[c]{0.5\textwidth}
      \centering
      \begin{tabular}{c}
        \small
        \lstinputlisting{examples/mult.k}
      \end{tabular}
    \end{subfigure}
    \hspace{-0.5cm}
    &
    \begin{subfigure}[c]{0.5\textwidth}
      \centering
      \begin{tabular}{c}
        \small
        \lstinputlisting{examples/multr.k}
      \end{tabular}
    \end{subfigure}
    \\
  \end{tabular}
  \vspace{0.1cm}
  \caption{Example model (left) and inferred types (right) for members
    of class \code{Person}.}  
  \label{fig:mult}
\end{figure*}

Each \code{Person} can have exactly one \code{mother}. This is
specified by line 4. No explicit multiplicity is specified, which
makes it a singleton.  A \code{Person} can also have many unique
\code{children}, which is specified by line 5 using the \code{Set}
collection. Line 6 specifies that a \code{Person} may have many
\code{cars}. It is written using a modifier and a multiplicity, which
semantically translates to a \code{Set} (\Klang{} default for a
multiplicity is \code{Bag}) of \code{Car}. Finally, a person may own
zero or more \code{prtflios}, where each entry itself is a \code{Set}
of \code{Stock}, specified to have a multiplicity of 0 or more. This
translates to \code{prtflios} being a \code{Bag} of \code{Set[Stock]}
with at least 1 entry and no upper limit.

\sysml{} models can also carry meta data information in them
(sometimes introduced by tools). To accommodate for this, \Klang{}
also provides the {\em annotation} construct. New annotations can be
created and applied to classes, expressions, functions etc. Currently,
each annotation has a name and a type.

\Klang{} also provides single line comments using `-{}-' at the
beginning of the line, and block comments using `===(=*)' as the token for
the beginning and the end of the comment.

\subsection{K Type Checking}

The \Klang{} type checker performs basic checks on the provided input
to ensure naming and type consistency. It is used to ensure that all
declarations, expressions, annotations etc. are logically sound and
reference names (functions, members, variables) that exist and are
type consistent in the given context. Type information for all
expressions and any other inferences made by the type checker are
saved and made available to all other analyses/modules in the \Klang{}
tool chain. Further, the type checker imposes a stricter set of rules
on the provided input to ensure that it can be completely and
correctly translated to SMT. More details are provided in
Section~\ref{sec:k2smt}. The type checker is implemented as a stand
alone module, which is invoked after the AST has been constructed by a
visitor (interfacing with ANTLR). The implementation is done using
Scala.

