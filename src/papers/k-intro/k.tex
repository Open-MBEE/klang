\section{K Syntax}

In this section we introduce the syntax of \Klang{}. We use the
\Klang{} example in Figure~\ref{fig:fs} as our running example for
discussing core concepts in \Klang{}. The example shows a basic model
of a flash file system modeled using \Klang{}. It is intended to be a
basis for discussing language features, and not a complete model of
the flash file system.

\begin{figure}
\centering
\begin{tabular}{c}
\hline \\
\lstinputlisting{examples/fs.k} \\ \\
\hline
\end{tabular}
\caption{A simple model of a spacecraft using \Klang{}.}
\label{fig:fs}
\end{figure}

\Klang{} is high level textual language which supports multiple
paradigms. It allows one to create \name{packages}, \name{classes},
and \name{functions}. Packages are collections of classes, which can
then be \name{imported} by other \Klang{} files. Line 1 in
Figure~\ref{fig:fs} shows an example of a package
declaration. Classes, similar to other languages provide a means for
abstracting and collecting properties. In \Klang{} classes may contain
functions, properties, and constraints (requirements). Lines 8 -- 11
in Figure~\ref{fig:fs} declare an \code{Entry} class, which contains
two members: property \code{name} of type \code{String}, and function
\code{size} that takes no arguments and returns an \code{Int}. For
function \code{size}, the function implementation is not specified for
function \code{size}. \code{String} is one of the six primitive types
provided by \Klang{}: \code{Int}, \code{Real}, \code{String},
\code{Char}, \code{Unit}, and \code{Boolean}. \Klang{} also provides
the following collections:

\begin{description}
\item [Bag:] collection of items not subject to any order
  or uniqueness constraints.
\item [Seq:] collection of items subject to an ordering, but
  no uniqueness constraints.
\item [Set:] collection of items subject to uniqueness
  constraints, but no ordering.
\end{description}

\Klang{} allows classes to inherit from one or more classes. For
example, class \code{Dir}, specified on lines 13 -- 18 extends the
\code{Entry} class. As with other languages, inheritance causes the
child classes to inherit the instance variables and functions of the
parent classes, but in addition, in \Klang{}, the child classes also
inherit the constraints from the parent classes. In the case of
multiple inheritance, \Klang{} requires that the property names be
unique across all classes. Functions on the other hand may be
overloaded by changing the function signature. Both class \code{File}
and \code{Dir} inherit from class \code{Entry} and define a body for
the \code{size} function. Lines 15 -- 17 in Figure~\ref{fig:fs}
provide the \code{size} function for the \code{Dir} class. It makes
use of the \code{sum} function, that is provided by \Klang{} for all
collections. This is the same function as declared in class
\code{Entry}. Currently, function bodies cannot be declared more than
once along an inheritance path. Functions may take an arbitrary number
of arguments and return a single value. \Klang{} also provides
\code{Tuple} to group objects together. On line 29, we see a
constraint being specified for class \code{Block} using the \code{req}
keyword. The constraint specifies that the size function of
\code{Block} should always return the same value as specified in the
global property \code{SIZE\_OF\_BLOCK} (left unspecified). Any number
of constraints can be specified at the global scope or within classes. 

\begin{enumerate}
\item Constructor with named arguments
\item Primitive types
\item Expressions
\item Function specifications
\item Quantification
\item Bounded types
\end{enumerate}

In class \code{FileSystem}

\subsection{K Type Checking}

Will use the same example as presented for the syntax.
