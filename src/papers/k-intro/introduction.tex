\section{INTRODUCTION}
\label{sec:introduction}

Modeling is the activity of formulating an abstract description of a
system to be implemented (or possibly already implemented). Modeling
includes such activities as requirements capture in the initial phases
and design of higher-level architectural decisions in later
stages. Modeling has been studied by various communities, of which at
least two can be identified: the model-based engineering community and
the formal methods community. The {\em model-based engineering}
community has suggested graphical modeling languages such as UML
\cite{uml} and SysML \cite{sysml}, a variant of UML.  Both have been
designed by the OMG (Object Management Group) technology standards
consortium. SysML is meant for systems development more broadly
considered, including physical systems as well as software systems, in
contrast to UML, which is mainly meant for software development. These
graphical formalisms have received a high degree of popularity in
industry due to their two dimensional format, also sometimes referred
to popularly as {\em boxology}: boxes and arrows. However, drawbacks
of these formalisms include lack of precise semantics, lack of
analysis capabilities, tedious GUI operations, requiring lots
of visual real estate even for simple models, as well as large volumes
of technologies. Learning UML and SysML is not just learning very
large languages, it is also learning a large set of additional tools
needed to work with models. We formulate the hypothesis that some of these
drawbacks in part are due to the lack of a simple textual language, at
a size comparable to a programming language, underlying the
graphical notations.

From an even earlier point in time, since the 1960s, the {\em
  formal methods} community, part of the computer science community,
and closely connected to the programming language community, has
proposed numerous formally founded specification languages. Several of
these are based on predicate logic and set theory. These languages
are, compared to UML and SysML, concise, small, well defined in the
form of semantics, and in recent time well supported with analysis
capabilities. The obvious observation is that it might be fruitful to
study the interaction between the two classes of formalisms. Consider
furthermore that programming languages are gaining in abstraction,
such as for example combining object-oriented and functional
programming. An example is Scala, which has many commonalities with very early formal
specification languages, such as for example VDM \cite{vdm78}, and
specifically its object-oriented variant VDM$^{++}$
\cite{vdmplusplus05}.  The study should therefore include the
interaction between the graphical formalisms, the formal methods
modeling languages, and programming languages.

We report on an effort to formally ground SysML in a textual formal
specification language, named K (Kernel language), designed
specifically for this purpose.  Our initial focus is on specifying and
analyzing class diagrams with constraints. K supports object-oriented
concepts such as classes, multiple inheritance, and object
instances. The contents of classes can be typed values, including
functions, and constraints over these expressed in higher-order
predicate logic.  K also contains programming constructs such as
variables, assignment statements, and looping constructs, and can as
such be seen as a wide-spectrum modeling and programming language.
The K language can be seen as a vehicle for giving semantics to SysML,
providing analysis capabilities, and even provide an alternative to
modeling with the mouse: writing textual models in K directly, just
like one normally writes programs.  We adhere to the school of thought
that modeling can be seen as programming in a language where some
parts of the model (program) at any point in time are executable, and
some maybe are not (yet).

The idea of merging modeling and programming in one language has been
suggested before, as will be discussed in the related work section.
Although K is not much different from previously suggested formalisms,  
our contribution is the creation of \Klang{} specifically in support of a
SysML model engineering tool set under development, to be used by
designers of the proposed 2022 mission to Jupiter's Moon Europa, also
referred to as the Europa Clipper mission concept
\cite{europa-clipper}.  The resulting tool set will support graphical
SysML modeling using MagicDraw \cite{magicdraw}, as well as
browser-based model viewing and editing, including of textual K
models. A first-order subset of K is furthermore translated to the input language of the
SMT-LIB standard, and currently processed with the Z3 SMT solver for
proving satisfiability of class definitions (are the constraints
consistent?), and model finding (find variable assignments satisfying
constraints, for example used in task scheduling).  The contribution here is
the handling of (multiple) class inheritance, which is typically not supported
by similar languages translated to SMT-LIB, as well as the allowance
of recursive class definitions. Multiple inheritance is a crucial part
of SysML, and therefore of K.

%The idea of merging modeling and programming in one language has been
%studied before, as will be discussed in the related work section. Our
%contribution is the development of \Klang{}, along with the technology
%to automatically translate it to SMT (for proving satisfiability) in
%the presence of multiple inheritance and constraints. This technology
%as part of a larger effort at NASA's Jet Propulsion Laboratory (JPL)
%to develop a SysML model engineering tool set to be used by designers
%of the proposed 2022 mission to Jupiter's Moon Europa, also referred to
%as the Europa Clipper mission concept \cite{europa-clipper}. This tool set,
%named EMS (Engineering Modeling System), supports graphical SysML
%modeling using MagicDraw \cite{magicdraw} as well as textual K
%modeling using any text editor of convenience. 

The paper is organized as follows. Section \ref{sec:k-syntax}
introduces a subset of the K language through an example, which is
similar to examples typically used to illustrate such formal
specification languages. Section \ref{sec:k2smt} outlines the
translation from K to the SMT-LIB input language for the purpose of
analysis of K models. This section is based on a different example
illustrating how K is actually currently used at JPL. Section
\ref{sec:usage} explains the integration of K within the SysML
development framework, as well as the usage of this. Section
\ref{sec:related-work} discusses related work, and finally Section
\ref{sec:conclusion} concludes the paper. 
